\documentclass[10pt]{article}
\usepackage{fontspec}
\usepackage{polyglossia}
\setdefaultlanguage{french}
\usepackage[a4paper,margin=1cm]{geometry}
\usepackage{url,hyperref}
\usepackage{siunitx}
\usepackage{schemabloc}
\usepackage{listings}
\usepackage{auto-pst-pdf}
\usepackage{pst-circ}
\usepackage{soul}
\usepackage{verbatim}
\usepackage{lmodern}
\usepackage{tikz}
\usepackage[european,cuteinductors,siunitx]{circuitikz}
\usepackage{xunicode,xltxtra}
\usepackage{graphicx}
\usepackage{amsmath}
\usepackage{minted}
\usepackage{multicol}
\title{\includegraphics{../../../images/inp-enseeiht} \\ ~ \\ ~ \\ ~ \\ ~ \\ BE Transistors }
\author{Ken Hasselmann, Guilhem Saurel}
\date{\today}
\begin{document}

 \section{Première Partie}
  \subsection{Faire une coupe technologique de ce transistor en tenant compte des paramètres précédents. Déterminer sans tenir compte des zones de charge d’espace $\beta_N$, $\beta_I$ et $I_S^\ast$ ainsi que la résistance série collecteur. En déduire le schéma électrique du transistor.}

    \begin{tabular}{|c|c|c|c|c|}
     \hline
      grandeur     & unité        & E (n)            & B (p)               & C (n)\\
     \hline                                                                
      $N$          &$(cm^{-3})$   & $2\cdot 10^{18}$ & $3\cdot 10^{61}$    & $9\cdot 10^{14}$ \\
     \hline                                                                
      $X$          & $(\mu m)$    & 1                & 1                   & 5 \\
     \hline                                                                
      $\mu_{min}$  & $(cm^2/V/s)$ & 125              & 1000                & 560 \\
     \hline                                                                
      $\mu_{MAJ}$  & $(cm^2/V/s)$ & 200              & 380                 & 1360 \\
     \hline                                                                
      $D_{min}$    & $(cm²/s)$    & 3.25             & 26                  & 14.6 \\
     \hline        
      $\tau_{min}$ & $(s)$        & $5\cdot 10^{-9}$ & $3.33\cdot 10^{-7}$ & $1.11\cdot 10^{-5}$\\
     \hline        
      $L_{min}$    & $(\mu m)$    & 1.27             & 29.4                & 127 \\
     \hline                                                                
      $X/L$        &              & 0.7              & 0.03                & 0.007 \\
     \hline                                                                
                   &              & normale          & courte              & courte\\
     \hline
    \end{tabular}
   
   $\gamma_E = 1 - \cfrac{D_P N_A L_n \tanh\cfrac{X_P}{L_n}}{D_N N_D L_p \tanh\cfrac{X_N}{L_p}} = 1 - \cfrac{D_P N_A X_P}{D_N N_D L_p \tanh\cfrac{X_N}{L_p}} \simeq 0.998$

   $\alpha^\ast = 1 - \cfrac{X_B^2}{2 L_n^2} \simeq 0.999$

   $\alpha_n = \gamma_E \cdot \alpha^\ast \simeq 0.997$

   $\beta_n = \cfrac{\alpha_n}{1-\alpha_n} = 354$

   $\gamma_C = \cfrac{1}{1 + \cfrac{D_{PC} N_{AB} X_B}{D_{NB} N_{DC} X_C}} \simeq 0.21$

   $\beta_I = \cfrac{\gamma_C}{1-\gamma_C} \simeq 0.27$

   $I_S^\ast = \alpha_n I_{ES} \simeq I_{ES} = q A_E n_i^2 \cfrac{D_N}{N_B X_B} = 1.77\cdot 10^{-15} A$

   $R_{sc} = \rho\cfrac{L}{S} = \cfrac{1}{q \mu_{EC} N_C}\cfrac{W_C}{A_C} = 124\Omega$

   On en déduit le schéma électrique du transistor :
 
  \begin{center}
   \begin{circuitikz} \draw
     (-1,2) node[anchor=east] {B}
      to[short, o-] (0,2)
      to[empty diode,color=red] (2,2) -- (6,2) 
      to[R=$R_{sc}$] (8,2)
      to[short, -o] (8,2) node[anchor=west, -o] {C}
     (0,2) to[empty diode,color=green] (0,0) -- (6,0) 
      to[controlled current source,i_=$I_B$,color=red] (6,2)
     (4,2) to[controlled current source,i^=$I_C$,color=green] (4,0)
     (3,0) to[short, -o] (3,-0.5) node[anchor=north] {E}
    ;
   \end{circuitikz}
  \end{center}

  \subsection{On polarise le transistor : $V_{CE}=8V$, $V_{BE}=0.68V$. Quel est le type de fonctionnement de ce transistor. En déduire son schéma équivalent. Calculer les paramètres électriques nécessaires en tenant compte de la polarisation. Comparer ces paramètres à la question 1. Conclusion}

   Ce transistor est en fonctionnement normal, vu que la jonction Émetteur/Base est polarisée en direct et que la jonction Base/Collecteur est polarisée en inverse. On ne garde donc que la partie verte du schéma précédent.

   \begin{center}
    \begin{circuitikz} \draw
      (-1,2) node[anchor=east] {B}
       to[short, o-] (0,2)
      (4,2) -- (6,2) 
       to[R=$R_{sc}$] (8,2)
       to[short, -o] (8,2) node[anchor=west, -o] {C}
      (0,2) to[empty diode,color=green] (0,0)
      (4,2) to[controlled current source,i^=$I_C$,color=green] (4,0) -- (0,0)
      (3,0) to[short, -o] (3,-0.5) node[anchor=north] {E}
     ;
    \end{circuitikz}
   \end{center}

   $\Phi_{EB} = U_T \ln\cfrac{N_E N_B}{n_i^2}$

   $\Phi_{BC} = U_T \ln\cfrac{N_B N_C}{n_i^2}$

   On part de l'hypothèse que la zone de charge d'espace de la jonction EB se situe entièrement dans la base.

   $\delta_{EB} = \sqrt{\cfrac{2\varepsilon}{qN_{EB}^\ast}\left(\Phi_{EB}-V_{BE}\right)}$, avec $N_{EB}^\ast \simeq N_B$

   Par contre, on ne peut pas faire cette hypothèse d'entrée pour la jonction BC, d'où 

   $\delta_{BC} = \delta_{BC_B} + \delta_{BC_C} = \sqrt{\cfrac{2\varepsilon}{qN_{BC}^\ast}\left(\Phi_{BC}+V_{CB}\right)}$, avec $N_{BC}^\ast \simeq N_C$ et $\delta_{BC_B} N_B = \delta_{BC_C} N_C$.

   Applications numériques :

   $\delta_{EB} = 8.73\cdot 10^{-2} \mu m$

   $\delta_{BC_B} = 9.84\cdot 10^{-2} \mu m$

   $\delta_{BC_C} = 3.28 \mu m$

   Maintenant, on peut trouver $W_B=X_B-\delta_{EB}-\delta_{BC_B} = 0.81\mu m$ et $W_C = X_B -\delta_{BC_C} = 1.7\mu m$, et on a toujours $W_E = X_E$. On peut donc remplacer tous les $X$ de la question précédente par ces $W$ :

   $\gamma_E = 1 - \cfrac{D_P N_A L_n \tanh\cfrac{W_P}{L_n}}{D_N N_D L_p \tanh\cfrac{W_N}{L_p}} = 1 - \cfrac{D_P N_A W_P}{D_N N_D L_p \tanh\cfrac{W_N}{L_p}} \simeq 0.998$

   $\alpha^\ast = 1 - \cfrac{W_B^2}{2 L_n^2} \simeq 1$

   $\alpha_n = \gamma_E \cdot \alpha^\ast \simeq 0.998$

   $\beta_n = \cfrac{\alpha_n}{1-\alpha_n} = 452$

   $\gamma_C = \cfrac{1}{1 + \cfrac{D_{PC} N_{AB} W_B}{D_{NB} N_{DC} W_C}} \simeq 0.10$

   $\beta_I = \cfrac{\gamma_C}{1-\gamma_C} \simeq 0.11$

   $I_S^\ast = \alpha_n I_{ES} \simeq I_{ES} = q A_E n_i^2 \cfrac{D_N}{N_B W_B} = 2.18\cdot 10^{-15} A$

   $R_{sc} = \rho\cfrac{L}{S} = \cfrac{1}{q \mu_{EC} N_C}\cfrac{W_C}{A_C} = 42.7\Omega$

   On en conclu que le $\beta_n$ est largement supérieur, car les longeurs électriques sont inférieures aux longueurs physiques : c'est l'effet Early

  \subsection{Calculer les courants Base et Collecteur dans les conditions de la question 2.} 

  $I_B = I_S^\ast \left(e^{\cfrac{V_{BE}}{U_T}}-1\right) = 0.498mA$

  $I_C = \beta_n I_B = 225mA$
  
  \subsection{Evaluer le paramètre d’effet Early μ puis la tension d’Early $V_A$. Faire le schéma électrique petit signal à fréquences moyennes puis à hautes fréquences dans les conditions de la question 2}

   $\mu = \cfrac{\partial V_{BE}}{\partial I_E}\cdot\cfrac{\partial I_E}{\partial W_B}\cdot\cfrac{\partial W_B}{\partial V_{CB}} = \cfrac{U_T}{I_E} \cdot\cfrac{-I_E}{W_B} \sqrt{\cfrac{\varepsilon N^\ast}{2qN_B^2\left(\Phi_{CB}+V_{CB}\right)}} = -1.97\cdot 10^{-4}$

   $r_c = \cfrac{U_T}{\mu I_E} = 588\Omega$

   $V_A = r_c I_C = -132V$
    
   Schéma à basses fréquences : on reprend le précédent, linéarise la diode et ajoute la résistance d'Early

   \begin{center}
    \begin{circuitikz} \draw
      (-1,2) node[anchor=east] {B}
       to[short, o-] (0,2)
      (4,2) -- (6,2) 
       to[R=$R_{sc}$] (8,2)
       to[short, -o] (8,2) node[anchor=west, -o] {C}
      (0,2) to[R=$r_b$] (0,0)
      (4,2) to[controlled current source,i^=$I_C$] (4,0) -- (0,0)
      (3,0) to[short, -o] (3,-0.5) node[anchor=north] {E}
      (6,2) to[R=$r_c$] (6,0) -- (4,0)
     ;
    \end{circuitikz}
   \end{center}

   Schéma à hautes fréquences : on reprend le précédent, et on ajoute les condensateurs correspondants aux ZCE.

   \begin{center}
    \begin{circuitikz} \draw
      (-1,2) node[anchor=east] {B}
       to[short, o-] (0,2)
      (4,2) -- (6,2) 
       to[R=$R_{sc}$] (8,2)
       to[short, -o] (8,2) node[anchor=west, -o] {C}
      (0,2) to[R=$r_b$] (0,0)
      (4,2) to[controlled current source,i^=$I_C$] (4,0) -- (0,0)
      (3,0) to[short, -o] (3,-0.5) node[anchor=north] {E}
      (6,2) to[R=$r_c$] (6,0) -- (4,0)
      (0,2) -- (2,2) to[C=$C_b$] (2,0)
      (2,2) to[C=$C_c$] (4,2)
     ;
    \end{circuitikz}
   \end{center}

  \subsection{On souhaite réaliser un transistor PNP ayant les mêmes dopages et dimensions. Calculer à partir de résultats de la question 1 les paramètres $\beta_N$ et $I_S^\ast$.}

  Nous avions trouvé les mobilités des majoritaires (au cas où...) à la question 1. Maintenant, il suffit de reprendre cette question en inversant les minoritaires et les majoritaires dans les mobilités :

   $\gamma_E = 1 - \cfrac{D_N N_D L_p \tanh\cfrac{X_N}{L_p}}{D_P N_A L_n \tanh\cfrac{X_P}{L_n}} = 1 - \cfrac{D_N N_D X_N}{D_P N_A L_n \tanh\cfrac{X_P}{L_n}} \simeq 0.991$

   $\alpha^\ast = 1 - \cfrac{X_B^2}{2 L_p^2} \simeq 0.998$

   $\alpha_n = \gamma_E \cdot \alpha^\ast \simeq 0.990$

   $\beta_n = \cfrac{\alpha_n}{1-\alpha_n} = 95$

   $I_S^\ast = \alpha_n I_{ES} \simeq I_{ES} = q A_E n_i^2 \cfrac{D_P}{N_B X_B} = 6.74\cdot 10^{-16} A$

   Sans surprises, les performances de ce PNP sont franchement moins bonnes que celles du même NPN, parceque la mobilité des électrons est supérieure à celle des trous.

 \section{Deuxième Partie}

  $R_B=12k\Omega$, $R_C = 1k\Omega$, $E=5V$

  \subsection{Trouver le point de fonctionnement pour}
   Pour ces trois question, on commence par faire deux lois des mailles : $E_B = V_{BE} + R_B I_B$ d'où $I_B = \cfrac{E_B -V_{BE}}{R_B}$, et $E=V_{CE} + R_C I_C$ d'où $I_C = \cfrac{E-V_{CE}}{R_C}$.
   \subsubsection{$E_B=0$}

    Dans ce cas, $V_{EB} = 0$ donc le transistor est bloqué.

   \subsubsection{$E_B=1V$}

    On est ici dans le cas étudié à l'exercice précédent : le transistor est en régime normal de fonctionnement.

    $I_B = \cfrac{E_B-V_{BE}}{R_B}$, où on prendra $V_{BE} = 0.6V$. $I_B = 33\mu A$. On en déduit $I_C = \beta_n I_B = 15mA$, or $R_C I_C > E$ donc on est en régime saturé. Donc en fait, $I_C = I_{C_{sat}} = \cfrac{E}{R_C+R_{SC}}$ où $R_{SC} \ll R_C$, donc $I_C = 5mA$, et $V_{CE} = R_{sc} I_C = 0.21V$.

   \subsubsection{$E_B=5V$}

    On est ici encore dans le cas étudié à l'exercice précédent : le transistor est en régime normal de fonctionnement.

    $I_B = \cfrac{E_B-V_{BE}}{R_B}$, où on prendra $V_{BE} = 0.6V$. $I_B = 367\mu A$. On en déduit $I_C = \beta_n I_B = 34.9mA$, or $R_C I_C > E$ donc on est en régime saturé. Donc en fait, $I_C = I_{C_{sat}} = \cfrac{E}{R_C+R_{SC}}$ où $R_{SC} \ll R_C$, donc $I_C = 5mA$, et $V_{CE} = R_{sc} I_C = 0.21V$.

  \subsection{Pour quelle valeur $E_B$ est-on à la limite du régime normal et du régime saturé ?}

   On attend cette limite pour $I_C = \beta_n I_B \simeq I_{C_{sat}}$, soit $I_B = \cfrac{I_{C_{sat}}}{\beta_n}$, et donc $E_B = V_{BE} +R_B I_B = 0.73V$. On vérifie bien que le transistor est saturé dans les deux cas précédents.

Diffusion des majoritaires de E dans B, conduction des minoritaires de B dans C


$$L_{p} = \sqrt{D_{p}\tau_{p}} --- \frac{D_{p}}{\mu_{p}} = \frac{kT}{q} = U_{T} --- \rho = \frac{1}{q(\bar{n}\mu_{n}+\bar{p}\mu_{p})} --- D=\mu U_{T} --- \delta=\sqrt{\frac{2\varepsilon}{qN^{*}}\left(\phi_{NP}-V\right)} --- \phi_{NP} = U_{T}\ln\left(\frac{N_{A}N_{D}}{n_{i}^{2}}\right) $$
$$I=I_{s}\left(e^{\frac{V}{U_{T}}}-1\right)$$
Direct : 
$$r_{d}=\frac{U_{T}}{I} --- C_{d} = \frac{IW_{n}^{2}}{2D_{p}U{T}} --- \tau = C_{d}r_{d} = \frac{W_{n}^{2}}{2D_{p}}$$
Inverse : 
$$C_{T} = \frac{\varepsilon A_{j}}{\delta} --- \alpha_{N}=\gamma_{E}\alpha^{*}=\left(1-\varepsilon_{1}\right)\left(1-\varepsilon_{2}\right)\cong 1-\varepsilon_{1}-\varepsilon_{2} --- \alpha_{I}=\gamma_{C}\alpha^{*} --- \gamma_{C} = \cfrac{1}{1+\cfrac{D_{NC}N_{D}W_{B}}{D_{PB}N_{AC}W_{C}}} --- I_{sn}=\frac{Aqn_{i}^{2}D_{n}}{N_{B}W_{B}}$$

$$
\begin{matrix}
&NPN&PNP&def\\
\gamma_{E}&\cfrac{I_{E_{N}}}{I_{E}}=\cfrac{1}{1+\cfrac{I_{S_{P}}}{I_{S_{N}}}}&\cfrac{I_{E_{P}}}{I_{E}}&\text{rendement d'injection de l'émetteur}\\
&1-\cfrac{D_{P_{E}}N_{A_{B}}L_{N_{B}}\tanh\cfrac{W_{P_{B}}}{L_{N_{B}}}}{D_{N_{B}}N_{D_{E}}L_{P_{E}}\tanh\cfrac{W_{N_{E}}}{L_{P_{E}}}}&1-\cfrac{D_{N_{E}}N_{D_{B}}L_{P_{B}}\tanh\cfrac{W_{B}}{L_{P_{B}}}}{D_{P_{B}}N_{A_{E}}L_{N_{E}}\tanh\cfrac{W_{E}}{L_{N_{E}}}}&\\
\gamma_{C}&1-\cfrac{D_{P_{C}}N_{A_{B}}L_{N_{B}}\tanh\cfrac{W_{P_{B}}}{L_{N_{B}}}}{D_{N_{B}}N_{D_{C}}L_{P_{C}}\tanh\cfrac{W_{N_{C}}}{L_{P_{C}}}}&1-\cfrac{D_{N_{C}}N_{D_{B}}L_{P_{B}}\tanh\cfrac{W_{B}}{L_{P_{B}}}}{D_{P_{B}}N_{A_{C}}L_{N_{C}}\tanh\cfrac{W_{C}}{L_{N_{C}}}}&\\
\alpha^{*}&1-\cfrac{W_{B}^{2}}{2L_{N}^{2}}&1-\cfrac{W_{B}^{2}}{2L_{N}^{2}}&\text{facteur de transport de la base}\\
\end{matrix}
$$

$$r_{b}=\cfrac{dV_{BE}}{dI_{B}} --- i_{C}=\beta_{N}i_{B}=\beta_{N}\cfrac{v_{B}}{r_{b}}=g_{m}v_{b}$$
Capacité de diffusion EB : $$C_{b}=\cfrac{I_{E}}{U_{T}}\cfrac{W^{2}}{2D}$$
Capacité de jonction CB : $$C_{c}=\cfrac{A_{c}}{\delta}$$
$$\delta=\sqrt{\cfrac{2\varepsilon}{qN^{*}}\left(\bar\phi_{CB}-V_{CB}\right)}$$
Résistance d'accès à la base : $$R_{BB'}=\frac{Y}{3\sigma WZ}$$
Facteur de réaction d'Early : $$\mu=\cfrac{\partial V_{BE}}{\partial V_{CB}}$$


\end{document}
