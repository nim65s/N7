\documentclass[11pt;a4paper;fleqn]{report}
\usepackage{circuitikz}
\usepackage{tikz}
\usepackage[utf8]{inputenc}
\usepackage{fontenc}
\usepackage[french]{babel}
\usepackage{lmodern}
\usepackage{amsmath}
\usepackage{amssymb}
\usepackage{mathrsfs}
\usepackage[top=2cm, bottom=2cm, left=2cm, right=2cm]{geometry}
\usepackage{multirow}
\usepackage{url,hyperref}
\usepackage{siunitx}
\usepackage{schemabloc}
\usepackage{listings}
\usepackage{auto-pst-pdf}
\usepackage{pst-circ}
\usepackage{soul}
\usepackage{verbatim}

\lstset {
 language=Matlab,
 basicstyle=\footnotesize,
 numbers=left,
 numberstyle=\footnotesize,
 stepnumber=1,
 numbersep=5pt,
 backgroundcolor=\color{white},
 showspaces=false,
 showstringspaces=true,
 showtabs=true,
 frame=single,
 tabsize=4,
 captionpos=b,
 breaklines=true,
 breakatwhitespace=true,
 title=\lstname,
}

\renewcommand{\thesection}{\thechapter .\alph{section}}

\title{\includegraphics{images/inp-enseeiht} \\ ~ \\ ~ \\ ~ \\ ~ \\ BE Fonction Hyperfréquences}
\author{Florian Kempenich, Guilhem Saurel}
\date{\oldstylenums{Vendredi 20 mai 2011}}

\begin{document}
 \begin{titlepage}
  \maketitle
 \end{titlepage}


% \tableofcontents

 \chapter{Étude de deux briques de base : tronçon de ligne et admittance parallèle}
  \section{}
   \begin{itemize}
    \item[•] \textit{Identifier la matrice \textbf{ABCD} du quadripole, puis déduire la matrice \textbf{abcd} en prenant comme impédance de référence des accès \textbf{$Z_c$}. Programmer sous MATLAB cette dernière en fonction de \textbf{$Z_c$}, \textbf{$v_p$}, \textbf{d}, et de la fréquence de travail \textbf{$f_0$} ; (\ul{valeurs initiales} : \textbf{$Z_c$} = $50\Omega$ ; \textbf{$v_p$} = $2E8m/s$ ; \textbf{d} = $2m$ ; \textbf{$f_0$} = $13.56MHz$).}

     \begin{center}
      \begin{pspicture}(7,3)
       \pnode(1,2.5){A}
       \pnode(1,0.5){B}
       \pnode(5,2.5){C}
       \pnode(5,0.5){D}
       \pnode(0,2.5){a}
       \pnode(0,0.5){b}
       \pnode(6,2.5){c}
       \pnode(6,0.5){d}
       \quadripole(A)(B)(C)(D){ABCD}
       \tension(B)(A){$V_1$}
       \tension[labeloffset=-0.5](D)(C){$V_2$}
       \wire[intensitylabel=$I_1$](a)(A)
       \wire(b)(B)
       \wire[intensitylabel=$I_2$,intensitylabeloffset=-0.5](c)(C)
       \wire(d)(D)
      \end{pspicture}
     \end{center}

     On sait que : 

     \[
      \begin{pmatrix}
       V_1 \\
       I_1
      \end{pmatrix}
      =
      \begin{pmatrix}
       A & B \\
       C & D
      \end{pmatrix}
      \begin{pmatrix}
       V_2 \\
       -I_2  
      \end{pmatrix}
     \]
  
     D'où

     \begin{eqnarray}
      V_1 &=& A V_2 - B I_2 \\
      I_1 &=& C V_2 - D I_2
     \end{eqnarray}

     Or, dans une ligne de transmissions, 

     \begin{eqnarray}
      V(z) &=& V^+ e^{+\gamma z} +V^- e^{-\gamma z} \\
      I(z) &=& \cfrac{V^+ e^{+\gamma z} -V^- e^{-\gamma z}}{Z_c}
     \end{eqnarray}

     Donc, dans notre cas, par application directe, et en prenant classiquement l'origine à l'utilisation :

     \begin{eqnarray}
      V_2 &=& V^+ + V^- \\
      V_1 &=& V^+ e^{+\gamma d} +V^- e^{-\gamma d} \\
      I_2 &=& \cfrac{V^+ -V^- }{Z_c}\\
      I_1 &=& \cfrac{V^+ e^{+\gamma d} -V^- e^{-\gamma d}}{Z_c}
     \end{eqnarray}

     On en déduit

     \begin{eqnarray}
      (1.5) \& (1.7) \Rightarrow V^- &=& \cfrac{Z_c I_2 + V_2}{2}
     \end{eqnarray}

     Donc, en injectant (1.9) dans (1.6), 

     \begin{eqnarray}
      V_1 &=& (V_2 - V^-) e^{+\gamma d} + V^- e^{-\gamma d} \nonumber \\
      &=& (V_2 - \cfrac{Z_c I_2 + V_2}{2}) e^{+\gamma d} + \cfrac{Z_c I_2 + V_2}{2} e^{-\gamma d} \nonumber \\
      &=& V_2 \left(\cfrac{e^{\gamma d} + e^{-\gamma d}}{2}\right) - I_2 Z_c \left(\cfrac{e^{\gamma d} - e^{-\gamma d}}{2}\right) \nonumber \\
      &=& V_2 \cosh(\gamma d) - I_2 Z_c \sinh(\gamma d)
     \end{eqnarray}

     D'où, par identification entre (1.10) et (1.1),

     \begin{eqnarray}
      A &=& \cosh(\gamma d) \\
      B &=& Z_c \sinh(\gamma d)
     \end{eqnarray}
      
     Et, de la même manière,

     \begin{eqnarray}
      C &=& Y_c \sinh(\gamma d) \\
      D &=& \cosh(\gamma d)
     \end{eqnarray}
      
     On en déduit la matrice \textbf{abcd} :
      
     \begin{eqnarray}
      a &=& A \\
      b &=& \cfrac{B}{Z_c} \\
      b &=& \cfrac{B}{Y_c} \\
      d &=& D
     \end{eqnarray}
      
     Ce qui donne, sous MATLAB :
     \lstinputlisting[language=Matlab,lastline=20]{sources/1a.m}

    \newpage
    \item[•] \textit{Déduire de la matrice \textbf{abcd} la matrice impédance réduite et la matrice S ; programmer ces relations de passage sous forme de fichier \textbf{*.m} ; vérifier la nature réciproque et sans perte de ce quadripole.}

     On sait que 
     \[
      \begin{pmatrix}
       v_1 \\
       v_1  
      \end{pmatrix}
      =
      \begin{pmatrix}
       z_{11} & z_{12} \\
       z_{21} & z_{22}
      \end{pmatrix}
      \begin{pmatrix}
       i_1 \\
       i_2
      \end{pmatrix}
     \]

     D'où

     \begin{eqnarray}
      v_1 &=& z_{11} i_1 + z_{12} i_2 \\
      v_2 &=& z_{21} i_1 + z_{22} i_2
     \end{eqnarray}

     Or, on rappelle (1.1) et (1.2) pour une impédance réduite :

     \begin{eqnarray}
      v_1 &=& a v_2 - b i_2 \\
      i_1 &=& c v_2 - d i_2
     \end{eqnarray}

     Donc

     \begin{eqnarray}
      (1.21) \Rightarrow c v_2 &=& i_1 + d i_2 \nonumber \\
      \Rightarrow v_2 &=& \cfrac{1}{c} i_1 + \cfrac{d}{c} i_2
     \end{eqnarray}

     D'où, par identification, entre (1.20) et (1.23), 

     \begin{eqnarray}
      z_{21} &=& \cfrac{1}{c} \\
      z_{22} &=& \cfrac{d}{c}
     \end{eqnarray}

     On remplace alors (1.23) dans (1.21) :

     \begin{eqnarray}
      v_1 &=& a \left( \cfrac{1}{c} i_1 + \cfrac{d}{c} i_2 \right) - b i_2 \nonumber \\
      &=& \cfrac{a}{c} i_1 + \cfrac{a d - b c}{c} i_2
     \end{eqnarray}

     Puis en identifiant (1.19) et (1.26) :

     \begin{eqnarray}
      z_{11} &=& \cfrac{a}{c} \\
      z_{12} &=& \cfrac{a d - b c}{c}
     \end{eqnarray}

     On en déduit les paramètres S :

     \begin{eqnarray}
      S_{11} &=& \cfrac{A + \cfrac{B}{Z_c} - C Z_c - D}{A+\cfrac{B}{Z_c} +C Z_c +D} \\
      S_{12} &=& 2 \cfrac{A D - B C}{A+\cfrac{B}{Z_c} +C Z_c +D} \\
      S_{21} &=& \cfrac{2}{A+\cfrac{B}{Z_c} +C Z_c +D} \\
      S_{22} &=& \cfrac{-A +\cfrac{B}{Z_c} + C Z_c +D}{A+\cfrac{B}{Z_c} +C Z_c +D}
     \end{eqnarray}
     
     On vérifie la nature réciproque de ce quadripole car le déterminant de la matrice S est égal à 1.

     De plus, cette matrice ne peut etre que sans pertes : en effet, jusqu'ici, tous les calculs ont été faits avec un certain $\gamma$, mais les données du problème ne nous permettent de déduire que $\gamma = j \beta = \cfrac{j \omega}{v_\varphi}$, comme est codée la variable "g" dans le fichier .m depuis le début.
     
     On programme ces relations de passage sous forme de fichier \textbf{*.m} :
     \lstinputlisting[language=Matlab,firstline=21,lastline=43,firstnumber=21]{sources/1a.m}


    \item[•] \textit{Montrer la variation en fonction de la fréquence de travail (\textbf{$f_0$} variant de $100MHz$ à $500MHz$)de la matrice \textbf{abcd} en traçant la partie réelle de \textbf{a} et la partie imaginaire de \textbf{b} ; conclure}.
     \lstinputlisting[language=Matlab,firstline=44,firstnumber=44]{sources/1a.m}

     \includegraphics[width=\linewidth,height=6cm]{images/1a}

   \end{itemize}

  \section{}
   \begin{itemize}
    \item[•] \textit{À l'aide de la matrice impédance, déterminer l'impédance vue par le générateur : $Z_E$}.

     \begin{center}
      \begin{pspicture}(7,3)
       \pnode(1,2.5){A}
       \pnode(1,0.5){B}
       \pnode(5,2.5){C}
       \pnode(5,0.5){D}
       \pnode(0,2.5){a}
       \pnode(0,0.5){b}
       \pnode(6,2.5){c}
       \pnode(6,0.5){d}
       \quadripole(A)(B)(C)(D){Z}
       \tension(B)(A){$V_1$}
       \tension[labeloffset=-0.5](D)(C){$V_2$}
       \wire[intensitylabel=$I_1$](a)(A)
       \wire(b)(B)
       \wire[intensitylabel=$I_2$,intensitylabeloffset=-0.5](c)(C)
       \wire(d)(D)
       \resistor[labeloffset=-1](d)(c){$Z_{ant}$}
      \end{pspicture}
     \end{center}

     Donc on a

     \begin{eqnarray}
      V_1 &=& Z_{11} I_1 + Z_{12} I_2 \\
      V_2 &=& Z_{21} I_1 + Z_{22} I_2 \\
      V_2 &=& - Z_{ant} I_2 \\
      Z_E &=& \cfrac{V_1}{I_1}
     \end{eqnarray}

     Donc en égalisant (1.34) et (1.35),

     \begin{eqnarray}
      -Z_{ant} I_2 &=& Z_{21} I_1 + Z_{22} I_2 \nonumber \\
      \Leftrightarrow -Z_{21} I_1 &=& \left(Z_{22} + Z_{ant}\right) I_2 \nonumber \\
      \Leftrightarrow I_2 &=& \cfrac{-Z_{21}}{Z_{22} + Z_{ant}} I_1
     \end{eqnarray}

     Donc en injectant (1.37) dans (1.33) de manière à identifier avec (1.36),

     \begin{eqnarray}
      V_1 &=& Z_{11} I_1 + Z_{12} \cfrac{-Z_{21}}{Z_{22} + Z_{ant}} I_1 \nonumber \\
      \Leftrightarrow Z_E &=& Z_{11} - \cfrac{Z_{12} Z_{21}}{Z_{22} + Z_{ant}}
     \end{eqnarray}

    \item[•] \textit{Déterminer la tension et le courant à l'entrée de la ligne : $V_1$, $I_1$ ; puis déduire la tension et le courant à la sortie de la ligne ($V_2$, $I_2$) à l'aide de la matrice \textbf{ABCD}.}
  
     Si la puissance dissipée est de $P_d = 33 dBm$, la puissance dissipée en Watts vaut $P_d = 10^{6.3}W$

     Si le générateur était chargé par $50 \Omega$, on aurait $P_d = \cfrac{|E_g|^2}{8R_c} \Rightarrow E_g = \sqrt{8 R_c Pd}$
     
     De plus 
     
     \begin{eqnarray}
      V_1 &=& \cfrac{E_g Z_E}{Z_E+R_c}
     \end{eqnarray}
     
     Donc 
     
     \begin{eqnarray}
      I_1 &=& \cfrac{V_1}{Z_E} = \cfrac{E_g}{Z_E+R_c}
     \end{eqnarray}
     
     Et, à l'aide de la matrice ABCD, on en déduit : 
     \[
      \begin{pmatrix}
       V_2 \\
       -I_2  
      \end{pmatrix}
      =
      \begin{pmatrix}
       A & B \\
       C & D
      \end{pmatrix}
      \begin{pmatrix}
       V_1 \\
       I_1
      \end{pmatrix}
     \]

     D'où

     \begin{eqnarray}
      V_2 &=& A V_1 + B I_1 \nonumber \\
      &=& \cfrac{E_g Z_E}{Z_E+R_c} \cosh(\gamma d) + \cfrac{E_g}{Z_E+R_c} Z_c \sinh(\gamma d)\\
      I_2 &=& -C V_1 - D I_1 \nonumber \\
      &=& - Y_c \cfrac{E_g Z_E}{Z_E+R_c} \sinh(\gamma d) - \cfrac{E_g}{Z_E+R_c} \cosh(\gamma d)
     \end{eqnarray}
  
    \item[•] \textit{Déduire $a_1$, $b_1$, $a_2$, $b_2$ à partir de la connaissance de $V_1$, $I_1$, $V_2$, $I_2$.}

     On a alors :
     
     \begin{eqnarray}
      a_1 = \cfrac{ V_1 + Z_c I_1}{2} \\
      b_1 = \cfrac{ V_1 - Z_c I_1}{2} \\
      a_2 = \cfrac{ V_2 + Z_c I_2}{2} \\
      b_2 = \cfrac{ V_2 - Z_c I_2}{2}
     \end{eqnarray}

    \item[•] \textit{Calculer les coefficients de réflexion à l'entrée et à la sortie de l'interconnexion, la puissance dissipée dans l'antenne ( correspondant à la puissance rayonnée), la puissance réfléchir vers le générateur, ainsi que la puissance développée par le générateur. Calculer les pertes d'insertion \textbf{IL} (\emph{Insertion Loss}) et les pertes en réflexions \textbf{RL} (\emph{Return Loss}) ; conclure.}
     
     Par définition, on a les coefficients de réflexion suivants :

     \begin{eqnarray}
      \Gamma_e &=& \cfrac{b_1}{a_1} \\
      \Gamma_s &=& \cfrac{b_2}{a_2}
     \end{eqnarray}

     Toujours par définition, la puissance dissipée dans l'antenne vaut :

     \begin{eqnarray}
      P_{ray} &=& \cfrac{V_2 I_2^*}{2}
     \end{eqnarray}

     La puissance réfléchie vers le générateur correspond à la puissance de l'onde $b_2$, qui est celle réfléchie vers le générateur :

     \begin{eqnarray}
      P_{ref} &=& \cfrac{|b_1|^2}{2}
     \end{eqnarray}

     La puissance développée par le générateur est donc logiquement la somme de la puissance rayonnée et de celle réfléchie vers le générateur, vu que la ligne de transmission est sans pertes :

     \begin{eqnarray}
      P_f &=&  \cfrac{V_2 I_2^* + |b_1|^2}{2}
     \end{eqnarray}

     Il ne reste plus qu'à calculer les pertes d'insertion et en réflexion :

     \begin{eqnarray}
      IL &=& 20 \log \cfrac{P_d}{P_f} \\
      RL &=& 20 \log \cfrac{P_d}{P_{ref}}
     \end{eqnarray}

   \end{itemize}

   Remarque : Dans cette partie, il n'y avait rien à faire en MATLAB, mais nous nous en sommes rendu compte qu'après l'avoir tout de même fait, donc voici le fichier, en bonus :
   \lstinputlisting[language=Matlab]{sources/1b.m}

  \section{}
   \begin{itemize}
    \item[•] \textit{Programmer sa matrice \textbf{abcd} ; (\ul{valeurs initiales} : une capacitance de $10pF$ en parallèle avec une self de $3.36nH$ et une résistance de $4k\Omega$ ; \textbf{$Z_c$} = $50\Omega$ ; \textbf{$f_0$} = $500MHz$)}
    \begin{eqnarray*}
     Y_p &=& 10*10^{-12} p + \cfrac{1}{3.36*10^{-9} p} + \cfrac{1}{4*10^3}
    \end{eqnarray*}

    Or la matrice d'un dipole $Y_p$ en parallèle est assez simple :

    \begin{center}
     \begin{pspicture}(3,3)
      \pnode(0,0.5){a}
      \pnode(0,2.5){b}
      \pnode(1,0.5){c}
      \pnode(1,2.5){d}
      \pnode(2,0.5){e}
      \pnode(2,2.5){f}
      \wire[intensitylabel=$I_1$](b)(d)
      \wire[intensitylabel=$I_2$,intensitylabeloffset=-0.5](f)(d)
      \wire(a)(e)
      \resistor[labeloffset=0](d)(c){$Y_p$}
      \tension[labeloffset=0.5](a)(b){$V_1$}
      \tension[labeloffset=-0.5](e)(f){$V_2$}
     \end{pspicture}
    \end{center}

    On a directement 

    \begin{eqnarray}
     V_1 &=& V_2 = \cfrac{I_1 + I_2}{Y_p}
    \end{eqnarray}

    D'où la représentation sous MATLAB suivante :

    \lstinputlisting[language=Matlab,lastline=8]{sources/1c.m}

    \item[•] \textit{En faisant varier la fréquence entre $500MHz$ et $1GHz$, calculer à l'aide du programme de passage développé en \emph{1.a} la variation du coefficient de réflexion et du coefficient de transmission en fonction de la fréquence ; trancer sous MATLAB la variation de $S_{11}$ à l'aide des commandes \textbf{plot} et \textbf{smith} ; tracer également la variation de $S_{21}$ à l'aide de \textbf{plot}.}

    Il nous faut la relation de mise en chaine de deux matrices \textbf{ABCD} : 

     \begin{center}
      \begin{pspicture}(13,3)
       \pnode(0,2.5){a}
       \pnode(0,0.5){b}
       \pnode(1,2.5){A}
       \pnode(1,0.5){B}
       \pnode(5,2.5){C}
       \pnode(5,0.5){D}
       \pnode(6,2.5){c}
       \pnode(6,0.5){d}
       \pnode(7,2.5){E}
       \pnode(7,0.5){F}
       \pnode(11,2.5){G}
       \pnode(11,0.5){H}
       \pnode(12,2.5){g}
       \pnode(12,0.5){h}
       \quadripole(A)(B)(C)(D){$A_1 B_1 C_1 D_1$}
       \quadripole(E)(F)(G)(h){$A_2 B_2 C_2 D_2$}
       \tension(B)(A){$V_1$}
       \tension[labeloffset=-0.5](d)(c){$V'$}
       \tension[labeloffset=-0.5](H)(G){$V_2$}
       \wire[intensitylabel=$I_1$](a)(A)
       \wire(b)(B)
       \wire[intensitylabel=$I'$,intensitylabeloffset=-0.5](c)(C)
       \wire(D)(F)
       \wire[intensitylabel=$-I'$](c)(E)
       \wire[intensitylabel=$I_2$,intensitylabeloffset=-0.5](g)(G)
       \wire(h)(H)
      \end{pspicture}
     \end{center}

     D'où

     \begin{eqnarray}
      V_1 &=& A_1 V' - B_1 I' \\
      I_1 &=& C_1 V' - D_1 I' \\
      V' &=& A_2 V_2 - B_2 I_2 \\
      -I' &=& C_2 V_2 - D_2 I_2
     \end{eqnarray}

     Donc, en substituant (1.56) et (1.57) dans (1.54) et (1.55),

     \begin{eqnarray}
      V_1 &=& A_1 A_2 V_2 - A_1 B_2 I_2 + B_1 C_2 V_2 - B_1 D_2 I_2 \nonumber \\
      &=& (A_1 A_2 + B_1 C_2) V_2 - (A_1 B_2 + B_1 D_2) I_2 \\
      I_1 &=& C_1 A_2 V_2 - C_1 B_2 I_2 + D_1 C_2 V_2 - D_1 D_2 I_2 \nonumber \\
      &=& (C_1 A_2 +D_1 C_2) V_2 - (C_1 B_2 + D_1 D_2) I_2
     \end{eqnarray}

    On programme donc la matrice \textbf{abcd} correspondant à la chaine des matrices du circuit d'adaptation et de la ligne de transmission, puis on traduit le tout en une matrice S qui dépend de la fréquence : 

    \newpage
    \lstinputlisting[language=Matlab,firstline=10,firstnumber=10]{sources/1c.m}

    \includegraphics[width=\linewidth,height=8cm]{images/1c}

   \end{itemize}
  \newpage
  \section{Modèle d'un stub court-circuité en parallèle}
   De la même manière, en remplaçant simplement $Y_p$ par $Z_c j \tan(\beta l)$

    \lstinputlisting[language=Matlab]{sources/1d.m}

    \includegraphics[width=\linewidth,height=8cm]{images/1d}

 \chapter{Amélioration du système d'antenne. Adaptation d'antenne}
  \section{Étude du circuit d'adaptation}
   Les deux solutions sont directement données par les applets java du site \verb|http://amanogawa.com| :
   
   \begin{center}
    \includegraphics[height=7cm]{images/s1}

    \includegraphics[height=7cm]{images/s2}
   \end{center}
\end{document}
