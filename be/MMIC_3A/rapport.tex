\documentclass{article}
\usepackage{fontspec}
\usepackage{polyglossia}
\setdefaultlanguage{french}
\usepackage[a4paper,margin=1cm]{geometry}

\usepackage{amsmath}
\usepackage{amssymb}
\usepackage{array}
\usepackage{auto-pst-pdf}
\usepackage{booktabs}
\usepackage{cite}
\usepackage{graphicx}
\usepackage{lmodern}
\usepackage{marvosym}
\usepackage{mathrsfs}
\usepackage{minted}
\usepackage{multicol}
\usepackage{multirow}
\usepackage{paralist}
\usepackage{schemabloc}
\usepackage{siunitx}
\usepackage{soul}
\usepackage{tikz}
\usepackage[european,cuteinductors,siunitx]{circuitikz}
\usepackage{url,hyperref}
\usepackage{verbatim}
\usepackage{xunicode,xltxtra}

\title{\includegraphics{../../../images/inp-enseeiht} \\ ~ \\ ~ \\ ~ \\ ~ \\ MMIC Tutorial}
\author{François Pierron \& Guilhem Saurel}
\date{\oldstylenums{\today}}

\begin{document}

\begin{titlepage}
    \setcounter{page}{0}
    \maketitle
    \thispagestyle{empty}
\end{titlepage}

\tableofcontents

On commence par trouver le point de polarisation. Pour cela, on se place sur la courbe IDS en fonction de VDS pour VGS = 0, ce qui donne un IDSS de 49mA, et on veut donc atteindre un IDS = IDSS/2 = 24.185 en modifiant VGS. On trouve alors un VGS de -0.26 V.

Pour la bobine, à 8GHz, S11 est optimal à 6.602 nH.
Si on ajoute la capa, S11 est minimal à 8GHz à 6.498 nH, et S13 est minimal à 8GHz pour 1.299 pF.
Cependant, vu que la capa depasse 1 pF, il faut changer son modèle.
idem, pour 6.497 nH et 3.525 PF.
Après avoir rajout quelques lignes et tee, 6.134 3.289
À ce moment là, on va voir sur le truc web, on dit qu’on a un spiral inductor de 6.134nH de 10um / 10um, et paf, la frequence de resonnance est de 8.56GHz, CEQUINEVAPASDUTOUT !
Du coup on s’en fout, on met deux bobines de 3.067

NB: c’est plus interessant de mettre les tee sur la couche IN. et OSEF si c’est marque que la couche BE fait 11 en vrai ça fait 10 à la fin du process techno.

Une fois qu’on a mis deux bobines, faut re-tune: 3.198 2.654 × 2

On fait alors le circuit complet. Ça se chevauche un poil.

On passe alors à la stabilite. C’est pas bon du tout, du coup on essaye avec des resistances au pif, sauf que ça rajoute trop de bruit.

On essaye alors plein de trucs au hasard pendant des heures.

\end{document}

