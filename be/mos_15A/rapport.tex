\documentclass[10pt]{article}
\usepackage{fontspec}
\usepackage{polyglossia}
\setdefaultlanguage{french}
\usepackage[a4paper,margin=2.5cm]{geometry}
\usepackage{url,hyperref}
\usepackage{siunitx}
\usepackage{schemabloc}
\usepackage{listings}
\usepackage{auto-pst-pdf}
\usepackage{pst-circ}
\usepackage{soul}
\usepackage{verbatim}
\usepackage{lmodern}
\usepackage{tikz}
\usepackage[european,cuteinductors,siunitx]{circuitikz}
\usepackage{xunicode,xltxtra}
\usepackage{graphicx}
\usepackage{amsmath}
\usepackage{minted}
\usepackage{multicol}
\title{\includegraphics{../../../images/inp-enseeiht} \\ ~ \\ ~ \\ ~ \\ ~ \\ BE Transistors MOS}
\author{Ken Hasselmann, Guilhem Saurel}
\date{\today}
\begin{document}

 \begin{titlepage}
  \maketitle
  \tableofcontents
 \end{titlepage}

 \section{La tension de seuil}
  \subsection{Calcul de la tension de seuil idéale $V_{T0}$.}
   $\phi_T = 2 U_T \ln\cfrac{N_A}{n_i} = 778$mV,
   et $\gamma = \cfrac{\sqrt{2q\varepsilon_SN}}{C_{ox}} = 0.569$\,;
   donc $V_{T0} = \phi_T + \gamma\sqrt{\phi_T} = 1.28$V.

  \subsection{Calcul des tensions de seuil réelles des transistors $V_{TN}$ et $V_{TP}$}
   Le transistor est un NMOS car le substrat est de type P.
   
   Donc  $V_{TP} = V_{TH} - |\phi_{MS}| - \delta{V_0}$. %TODO : démarche

   
   Or $\phi_{MS} = W_M - W_S = W_M - \left(\chi + \cfrac{E_g}{2q} + U_T ln \cfrac{N_a}{ni} \right) = \phi_{MI} - U_T ln \left(\cfrac{N_a}{ni}\right) = -949$mV (ce que nous retrouvons d’ailleurs sur l’abaque …),
   et $\Delta V_0 = \cfrac{Q_0}{C_{ox}} = \cfrac{q N_{ss} t_{ox}}{\varepsilon_{ox}} = 7.2$mV.
   
   Donc $V_{TP} = 324$mV.
      
   Si on prend un PMOS (substrat N) on aura :
   $V_{TN} = V_{TH} + |\phi_{MS}| + |\Delta{V_0}| = 2.24$V.

  \subsection{Types de charges à implémenter}
  Dans le cas du transistor NMOS, il faut diminuer la tension de seuil. 
  Il faut alors implanter des charges de type opposé au substrat (des donneurs),
  dans une quantitée 
  $N_A \delta(x) = \cfrac{\varepsilon_{ox} \Delta V}{t_{ox} q} = 5.23\cdot 10^{11}$.

  Dans le cas du transistor PMOS, il faut augmenter la tension de seuil. 
  Il faut alors implanter des charges de type opposé au substrat (des accepteurs),
  dans une quantitée  
  $N_A \delta(x) = \cfrac{\varepsilon_{ox} \Delta V}{t_{ox} q} = 2.13\cdot 10^{12}$.
  
 
 \section{Caractéristiques C(v)}
  En accumulation, C(v) est constante, égale à $C_{ox}$, puiqu’il n’y a pas de ZCE dans le semi-conducteur, donc $C=C_{ox}$.

  En dépeuplement, une ZCE dans le semi-conducteur apparait progressivement, donc $C=\cfrac{C_{ox} C_{si}}{C_{ox}+C_{si}}=
  \cfrac{C_{ox}}{1+\cfrac{C_{ox}}{C_{si}}}=
  \cfrac{C_{ox}}{1+\cfrac{\varepsilon_{ox}\delta_{si}}{t_{ox}\varepsilon_{si}}}=
  \cfrac{C_{ox}}{1+\cfrac{t_{ox}}{\varepsilon_{ox}}\sqrt{2qN\varepsilon_{si}}}=
  \cfrac{C_{ox}}{\sqrt{1+\cfrac{4V}{\gamma^2}}}$
 
  \newpage
 
 \section{Polarisation d’un Transistor}
 
 \subsection{$V_{IN1} = 5V$ et $V_{IN2} = 0V$}
 
 Si $V_{IN1} = 5$V et $V_{IN2} = 0$V alors $V_{GS1} = V_{TH} < 5$V, donc le transistor T1 est passant,
 et $V_{GS2} = V_{TH} > 0$V, donc le transistor T2 est bloqué.
 
 Si on considère $V_{OUT} = 2$V, on a : $I_{D} = \cfrac{V_{CC} - V_{OUT}}{R}$.
 Or dans ce cas $V_{GD1} > V_{TH}$, donc on est en zone ohmique, d'où
 $I_D = \mu_0 C_{ox} \cfrac{W}{L} \left(V_G - V_{TH} - \cfrac{V_D}{2}\right)V_D$
 donc $\cfrac{V_{CC} - V_{OUT}}{R} = \mu_0 C_{ox} \cfrac{W}{L} \left(V_G - V_{TH} - \cfrac{V_D}{2}\right)V_D$ avec $V_D = V_{OUT}$,
 donc $R =\cfrac{V_{CC} - V_{OUT}}{\mu_0 C_{ox} \cfrac{W}{L} \left(V_G - V_{TH} - \cfrac{V_{OUT}}{2}\right)V_{OUT}} = 5.84$k$\Omega$.
 
 \subsection{$V_{IN1} = V_{IN2} = 5V$}
 
 Si $V_{IN1} = 5$V et $V_{IN2} = 5$V alors $V_{GS1} = V_{GS2} = V_{TH} < 5$V  donc les transistors sont passant. Ils sont équivalents donc le même courant les traverse.
 Si on considère qu'ils sont de plus tous deux en zone ohmique alors :
 $I_{D} = \cfrac{V_{CC} - V_{OUT}}{2 R} = \mu_0 C_{ox} \cfrac{W}{L} \left(V_G - V_{TH} - \cfrac{V_D}{2}\right)V_D$
 avec $V_D = V_{OUT}$ donc
 $I_D = \mu_0 C_{ox} \cfrac{W}{L} \left(V_G - V_{TH} - \cfrac{V_{OUT}}{2}\right)V_{OUT}$
 De plus, $V_{OUT} = V_{CC} - 2 R I_D$
 Finalement, $V_{OUT} = V_{CC} - 2 R \mu_0 C_{ox} \cfrac{W}{L} \left(V_G - V_{TH} - \cfrac{V_{OUT}}{2}\right)V_{OUT} = 1.14$V.
 
 \subsection{$V_{IN1} = V_{IN2} = V_0 = 1V$}
 Si $V_{IN1} = V_{IN2} = 1$V alors $V_{GS1} = V_{GS2} = V_{TH} < 1$V donc les transistors sont passant. Mais par contre les transistors seront en zone saturée car au vu de la réponse à la question précédente on aura $V_{GD} < V_{TH}$. En revanche les courants dans les deux transistors restent les mêmes.
 $I_D = \cfrac{V_{CC} - V_{OUT}}{2 R} = \mu_0 C_{ox} \cfrac{W}{2L}{\left(V_G - V_{TH}\right)}^2 \Rightarrow
 V_{OUT} = V_{CC} - 2 R I_D = V_{CC} - 2 R \mu_0 C_{ox} \cfrac{W}{2L}{\left(V_G - V_{TH}\right)}^2 = 4.96$V.
 
 \section{Analyse petit signal}
  \subsection{Schéma électrique}
   \begin{center}
    \begin{circuitikz} \draw
     (0,2) to[short, o-] (1,2) to[R=$R_G$] (1,0)
     (0,0) -- (5,0) to[R=$R_S$] (5,2)
     (3,0) to[R=$R_D$] (3,2) 
      to[controlled current source=$I_D$] (5,2)
      to[short, -o] (6,2)
     (5,0) -- (6,0)
     ;
    \end{circuitikz}
   \end{center}
  
  Où $I_D = \mu_0 C_{ox} \cfrac{W}{2L} (V_G - V_{TH})^2$

 
 \appendix
 \section{Fichier Scilab}
 Voici le fichier scilab qui contient toutes nos applications numériques\,:
 
 \inputminted[linenos]{matlab}{scilab.sci}

\end{document}
