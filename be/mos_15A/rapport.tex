\documentclass[10pt]{article}
\usepackage{fontspec}
\usepackage{polyglossia}
\setdefaultlanguage{french}
\usepackage[a4paper,margin=2.5cm]{geometry}
\usepackage{url,hyperref}
\usepackage{siunitx}
\usepackage{schemabloc}
\usepackage{listings}
\usepackage{auto-pst-pdf}
\usepackage{pst-circ}
\usepackage{soul}
\usepackage{verbatim}
\usepackage{lmodern}
\usepackage{tikz}
\usepackage[european,cuteinductors,siunitx]{circuitikz}
\usepackage{xunicode,xltxtra}
\usepackage{graphicx}
\usepackage{amsmath}
\usepackage{minted}
\usepackage{multicol}
\title{\includegraphics{inp-enseeiht} \\ ~ \\ ~ \\ ~ \\ ~ \\ BE Transistors MOS}
\author{Ken Hasselmann, Guilhem Saurel}
\date{\today}
\begin{document}

 \begin{titlepage}
  \maketitle
  \tableofcontents
 \end{titlepage}

 \section{La tension de seuil}
  \subsection{Calcul de la tension de seuil idéale $V_{T0}$.}
   $\phi_T = 2 U_T \ln\cfrac{N_A}{n_i} = 778$mV,
   et $\gamma = \cfrac{\sqrt{2q\varepsilon_SN}}{C_{ox}} = 0.569$\,;
   donc $V_{T0} = \phi_T + \gamma\sqrt{\phi_T} = 1.28$V.

  \subsection{Calcul des tensions de seuil réelles des transistors $V_{TN}$ et $V_{TP}$}
   Le transistor est un NMOS car le substrat est de type P.
   
   Donc  $V_{TP} = V_{TH} - |\phi_{MS}| - \delta{V_0}$. %TODO : démarche

   
   Or $\phi_{MS} = W_M - W_S = W_M - \left(\chi + \cfrac{E_g}{2q} + U_T ln \cfrac{N_a}{ni} \right) = \phi_{MI} - U_T ln \left(\cfrac{N_a}{ni}\right) = -949$mV (ce que nous retrouvons d’ailleurs sur l’abaque …),
   et $\Delta V_0 = \cfrac{Q_0}{C_{ox}} = \cfrac{q N_{ss} t_{ox}}{\varepsilon_{ox}} = 7.2$mV.
$I_D = \cfrac{V_{CC} - V_{OUT}}{2 R} = \mu_0 C_{ox} \cfrac{W}{2L}{\left(V_G - V_{TH}\right)}^2 \Rightarrow
 V_{OUT} = V_{CC} - 2 R I_D = V_{CC} - 2 R \mu_0 C_{ox} \cfrac{W}{2L}{\left(V_G - V_{TH}\right)}^2 = 4.96$V.
 
 \section{Analyse petit signal}

 \appendix
 \section{Fichier Scilab}
 Voici le fichier scilab qui contient toutes nos applications numériques\,:
 
 \inputminted[linenos]{matlab}{scilab.sci}

\end{document}
