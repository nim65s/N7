\documentclass{article}
\usepackage{fontspec}
\usepackage{polyglossia}
\setdefaultlanguage{french}
\usepackage[a4paper,margin=1cm]{geometry}

\usepackage{amsmath}
\usepackage{amssymb}
\usepackage{array}
\usepackage{auto-pst-pdf}
\usepackage{booktabs}
\usepackage{cite}
\usepackage{graphicx}
\usepackage{lmodern}
\usepackage{marvosym}
\usepackage{mathrsfs}
\usepackage{minted}
\usepackage{multicol}
\usepackage{multirow}
\usepackage{paralist}
\usepackage{schemabloc}
\usepackage{siunitx}
\usepackage{soul}
\usepackage{tikz}
\usepackage[european,cuteinductors,siunitx]{circuitikz}
\usepackage{url,hyperref}
\usepackage{verbatim}
\usepackage{xunicode,xltxtra}

\title{\includegraphics{../../../images/inp-enseeiht} \\ ~ \\ ~ \\ ~ \\ ~ \\ Conception of Analog Circuits}
\author{François Pierron \& Guilhem Saurel}
\date{\oldstylenums{\today}}

\begin{document}

\begin{titlepage}
    \setcounter{page}{0}
    \maketitle
    \vfill
    \tableofcontents
    \thispagestyle{empty}
\end{titlepage}

VCO
===
On choisi une architecture avec deux opamps, parce qu’on travaille à basse fréquence (<500Hz)

Du coup, il faut un 

OPAMP
-----

Source de courant: 20mA. la tension de bias est amenée par le niveau hiérarchique supérieur, vu qu’elle sera partagée entre tous les 
    opamp de la PLL. Pour améliorer la qualité de la recopie de courant, il faut que les transistors de recopie aient des L assez grands,
    vu que c’est le L qui drive la pente de la caractéristique.
plage de sortie: quelque chose de plus grand que 2.5V, pour pouvoir réduire après. On met donc plus de courant dans la source, via un 
    second transistor
on tune les W & L des transistors pour tenter d’avoir une tension de sortie moyenne d’environs 2.5V
    et que lorsqu’on reboucle, la tension sur les drains de la charge active soient proches
La première fréquence de coupure est liée à la capa, donc il faut qu’elle soit suffisament faible pour que cette fréquence soit >400Hz. Non, en fait, on veut que le produit gain-bande soit suffisant., parce que peut être qu’on travaillera à un autre gain que le gain max. Par contre, faut faire un compromis pour garder une marge de phase >60°

MaxGain: 91
MargePhase: 69
MargeGain 10
GainBande: 7.5MHz
Offset : 0.4uA

Reste du VCO
------------

ben, ça juste marche. juste pour la simu, c’est un peu tendu, faut faire un paramétrique sur la tension d’entrée, ploter en fonction de ça, et afficher 1/Period.
Du coup on se rend compte que la tension d’entrée correspondant à une fréquence de sortie de 400Hz est de 1.6, donc on augmente la capa juqu’à 10n pour avoir 400Hz @ 2.5V

Diviseur
========

on met 3 bascules D

Bascule D
---------

On met deux NAND pour faire une RS, et en entrée il faut deux OR dont une a une entrée inversée.

bascule inverseuse
------------------
le P a un W/L trois fois plus gros



\end{document}
