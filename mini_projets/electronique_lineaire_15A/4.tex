 \chapter{Applications aux différents étages}
  \section{Utilisation d’un algorithme itératif}
   Une fois un programme de calcul des caractéristiques d’un étage codé, 
   il nous est venu l’idée de développer un peu ce programme pour qu’il trouve
   tout seul les valeurs des différents composants.

   ~

   Le principe est simple: pour un collecteur commun, par exemple, on travaille
   dans un espace à 4 dimensions où les dimensions représentent les résistances
   $R_{B1}$, $R_{B2}$, $R_{E1}$ et $R_{E2}$ (où, bien entendu, chacune peut prendre
   les valeurs de la série E12, entre 10 \ohm et 1M\ohm).
   On part d’un point de cet espace qui semble cohérent, et on recalcule les
   caractéristiques du circuit pour chaque point adjacent. On regarde ensuite 
   quel est le point valide le mieux le cahier des charges, et on recommence à partir
   de ce dernier.

   Une fois que nous avons trouvé un point qui correspond, on cherche encore un peu
   pour voir s’il n’y en aurait pas d’autres, et on prend ensuite le plus performant.
   Lorsque l’on a encore l’embarras du choix, on prend les résistances les plus 
   importantes pour que les courants soient les plus faibles possibles.

   ~
   
   Il est alors intéressant d’utiliser l’outil informatique pour prendre en compte
   facilement le fait que le $\beta$ varie du simple au double, voire que les 
   résistances de la série E12 sont données à 10\%.

   Les scripts qui implémentent cet algorithme en python sont disponibles à
   cette adresse: 
   
   https://github.com/nim65s/N7/tree/master/mini\_projets/electronique\_lineaire\_15A


  \section{Collecteur Commun du dernier étage}
   Pour tester ce script, le plus simple est de commencer par le dernier étage.

   En effet, on sait que sa résistance de sortie est très importante, et que sa
   dynamique de sortie doit tenir les 6V.

    Par contre, pour ces calculs, la résistance de sortie du troisième étage intervient.
    On peut en général la négliger, mais puisque c’est une machine qui calcule, ça ne
    change pas grand chose. Nous avons donc estimé la résistance de sortie du troisième
    étage à 5.6\kilo\ohm, puisque c’est $R_C$ de la paire différentielle.

   On trouve les caractéristiques suivantes:

   \begin{tabular}{|c|c|c|}
    \hline
     Grandeur  & Idéale           &   Résistances connues à 10\% \\\hline
     $\beta$     & 400.00 à 800.00 & 400.00 à 800.00 \\\hline
     $R_{B1}$   &      10.00k      & 9.00k à 11.00k  \\\hline
     $R_{B2}$   &     680.00k      &612.00k à 748.00k\\\hline
     $R_{E1}$   &      100.00      & 90.00 à 110.00  \\\hline
     $R_{E2}$   &      820.00      & 738.00 à 902.00 \\\hline
     $R_C$       &      100.00      & 90.00 à 110.00  \\\hline
     $R_G$       &      5.60k       &      5.60k       \\\hline
     $Z_L$       &      5.00k       &      5.00k       \\\hline
     $R_B$       &      9.86k       & 7.26k à 13.25k  \\\hline
     $R_E$       &      920.00      & 828.00 à 1.01k  \\\hline
     $E_B$       &      11.83       &      11.79       \\\hline
     $I_C$       & 11.88m à 12.04m & 10.71m à 13.45m \\\hline
     $g_m$       &457.08m à 463.12m&411.73m à 517.25m\\\hline
     $r_b$       & 863.71 à 1.75k  & 773.32 à 1.94k  \\\hline
     $Z_E$       & 4.85k à 19.11k  & 2.82k à 32.55k  \\\hline
     $Z_S$       & 92.73 à 100.38  & 67.03 à 140.22  \\\hline
     $Av$        &       0.99       &       0.99       \\\hline
     $DS$         &  16.74 à 16.97  &  13.39 à 21.14  \\\hline
  \end{tabular}


  On peut également vérifier que la dynamique de sortie pour 100\ohm{}
   reste suffisament bonne:


   \begin{tabular}{|c|c|c|}
    \hline
    Grandeur  & Idéale           &   Résistances connues à 10\% \\\hline
    $\beta$   & 400.00 à 800.00  & 400.00 à 800.00 \\\hline
    $R_{B1}$  &      10.00k      & 9.00k à 11.00k  \\\hline
    $R_{B2}$  &     680.00k      &612.00k à 748.00k\\\hline
    $R_{E1}$  &      100.00      & 90.00 à 110.00  \\\hline
    $R_{E2}$  &      820.00      & 738.00 à 902.00 \\\hline
    $R_C$     &      100.00      & 90.00 à 110.00  \\\hline
    $R_G$     &      5.60k       &      5.60k       \\\hline
    $Z_L$     &      100.00      &      100.00      \\\hline
    $R_B$     &      9.86k       & 7.26k à 13.25k  \\\hline
    $R_E$     &      920.00      & 828.00 à 1.01k  \\\hline
    $E_B$     &      11.83       &      11.79       \\\hline
    $I_C$     & 11.88m à 12.04m  & 10.71m à 13.45m \\\hline
    $g_m$     &457.08m à 463.12m &411.73m à 517.25m\\\hline
    $r_b$     & 863.71 à 1.75k   & 773.32 à 1.94k  \\\hline
    $Z_E$     & 4.35k à 15.56k   & 2.16k à 31.25k  \\\hline
    $Z_S$     & 92.73 à 100.38   & 67.03 à 140.22  \\\hline
    $Av$      &   0.95 à 0.99    &   0.94 à 0.99   \\\hline
    $DS$      &   2.12 à 2.15    &   1.58 à 2.90\\\hline
   \end{tabular}


   On remarque que les calculs prenant en compte les résistances connues à 10\% 
   ne sont pas satisfaisants. Cependant, ils sont donnés pour le pire des cas,
   c’est à dire le pire cas cumulé pour toutes lés résistances et le $\beta$, dans chaque calcul
   d’une des caractéristiques du circuit, ce qui a peu de chances d’arriver.
    
  \section{Collecteur Commun du premier étage}
   L’étage facile à calculer suivant est le premier, vu que sa résistance d’entrée
   nous est imposée.
   Une fois de plus, nous devons estimer l’impédance d’entrée de l’étage suivant ; 
   nous avons pris 15\kilo\ohm.

   Les résultats sont les suivants:

   \begin{tabular}{|c|c|c|}
    \hline
 Grandeur  &       Idéale    &  Résistances connues à 10\% \\\hline
 $\beta$   & 400.00 à 800.00 & 400.00 à 800.00 \\\hline
 $R_{B1}$  &     560.00k      &504.00k à 616.00k\\\hline
 $R_{B2}$  &      1.00M       & 900.00k à 1.10M \\\hline
 $R_{E1}$  &      27.00       &  24.30 à 29.70  \\\hline
 $R_{E2}$  &      15.00       &  13.50 à 16.50  \\\hline
 $R_C$     &      100.00      & 90.00 à 110.00  \\\hline
 $R_G$     &      50.00       &      50.00       \\\hline
 $Z_L$     &      15.00k      &      15.00k      \\\hline
 $R_B$     &     358.97k      &264.34k à 482.62k\\\hline
 $R_E$     &      42.00       &  37.80 à 46.20  \\\hline
 $E_B$     &       7.69       &   7.12 à 8.23   \\\hline
 $I_C$     & 7.55m à 14.45m  & 5.21m à 20.72m  \\\hline
 $g_m$     &290.37m à 555.88m&200.30m à 796.91m\\\hline
 $r_b$     & 719.58 à 2.76k  & 501.94 à 3.99k  \\\hline
 $Z_E$     & 15.87k à 34.58k & 7.86k à 70.45k  \\\hline
 $Z_S$     &  8.56 à 11.88   &  5.97 à 17.12   \\\hline
 $Av$      & 858.77m à 0.98  & 790.53m à 0.99  \\\hline
 $DS$      &226.26m à 433.15m&140.45m à 683.14m\\\hline
\end{tabular}

    Ici, même dans le cas idéal, la variation de $\beta$ entraine un dépassement des
    limites authorisées pour $Z_E$… Il nous faudra donc probablement aviser une fois 
    sous Spice.

  \section{Émetteur Commun du second étage}

  Ce qui nous intéresse pour cet étage est le gain de la dynamique de sortie.
  Voici un circuit qui devrait aller:
   \begin{tabular}{|c|c|c|}
    \hline
 Grandeur  & Idéale    &  Résistances connues à 10\% \\\hline
 $\beta$   &         400.00 à 800.00 &  400.00 à 800.00 \\\hline
 $R_{B1}$  &          22.00k     &   19.80k à 24.20k \\\hline
 $R_{B2}$  &          22.00k     &   19.80k à 24.20k \\\hline
 $R_E$  &          47.00k     &   42.30k à 51.70k \\\hline
 $R_C$  &          12.00k     &   10.80k à 13.20k \\\hline
 $Z_L$  &          16.00k     &        16.00k      \\\hline
 $R_B$  &          11.00k     &   8.10k à 14.79k  \\\hline
 $E_B$  &           6.00      &     5.40 à 6.60   \\\hline
 $I_C$  &          0.11m      &   92.78u à 0.14m  \\\hline
 $g_m$  &          4.42m      &    3.57m à 5.45m  \\\hline
 $r_b$  &    90.54k à 181.14k & 73.34k à 224.19k \\\hline
 $Z_E$  &     5.18k à 19.62k  &  2.49k à 40.71k  \\\hline
 $Z_S$  &          12.00k     &   10.80k à 13.20k \\\hline
 $Av$   &          30.28      &    21.12 à 42.98  \\\hline
 $DS$   &           1.57      &     1.10 à 2.24   \\\hline
\end{tabular}
