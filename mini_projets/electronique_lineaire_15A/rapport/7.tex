\chapter{Montage sur plaquette labdec et tests}
\section{Première séance: deux premiers étages}
\subsection{Valeur des résistances}
\begin{tabular}{|l|c|c|}
    \hline
    Résistance & théorique & expérimentale \\
    \hline
    $R_{1b1}$ & 560k & 596k \\
    \hline
    $R_{1b2}$ & 1M & 1.013M \\
    \hline
    $R_{1c}$ & 100 & 99.9 \\
    \hline
    $R_{1e}$ & 100 & 99.9 \\
    \hline
    \hline
    $R_{2b}$ & 22k & 22k et 21.9k \\
    \hline
    $R_{2c}$ & 10k & 9.77k \\
    \hline
    $R_{2e1}$ & 82 & 81.3 \\
    \hline
    $R_{2e2}$ & 47k & 46.7k \\
    \hline
\end{tabular}

\subsection{Tensions de polarisation}
\begin{tabular}{|l|r|r|r|r|}
    \hline
    & Étage 1: théorique & Étage 1: expérimental & Étage 2: théorique & Étage 2: expérimental \\
    \hline
    c & 11.41 & 11.14 & 10.86 & 11.16 \\
    \hline
    b & 1.28 & 1.60 & 5.99 & 4.94 \\
    \hline
    e & 0.59 & 0.96 & 5.40 & 4.47 \\
    \hline
\end{tabular}

\subsection{Tension d’alimentation}
$V_{CC}$ = 12.09V.

\subsection{Amplitudes des tensions}
\begin{tabular}{|l|r|}
    \hline
    entrée & 6.37m \\
    \hline
    sortie du premier étage & 6.30m \\
    \hline
    sortie du second étage & 143.00m \\
    \hline
\end{tabular}

\subsection{Courant total}
Nos deux premiers étages consomment 9.85mA.

\subsection{Fréquences de coupure}

\section{Seconde séance: deux derniers étages}

R

2177
5538
5585
3911
3922
3929
3981

99.63
68.8
993.8
329000
672000

V

11.99
-12.05

11.055
1.577
0.936

10.85
5.982
5.413

6.669 7.067
-6.62m -8.27m
-0.6167
-4.084
-4.698

11.476
6.181
5.635
5.278

FCBF 98
FCHF 190k

Ze: 68000
Zs
