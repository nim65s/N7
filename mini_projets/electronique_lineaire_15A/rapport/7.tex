\chapter{Montage sur plaquette labdec et tests}
\section{Première séance: deux premiers étages}
\subsection{Valeur des résistances}
\begin{tabular}{|l|c|c|}
    \hline
    Résistance & théorique & expérimentale \\
    \hline
    $R_{1b1}$ & 560k & 596k \\
    \hline
    $R_{1b2}$ & 1M & 1.013M \\
    \hline
    $R_{1c}$ & 100 & 99.9 \\
    \hline
    $R_{1e}$ & 100 & 99.9 \\
    \hline
    \hline
    $R_{2b}$ & 22k & 22k et 21.9k \\
    \hline
    $R_{2c}$ & 10k & 9.77k \\
    \hline
    $R_{2e1}$ & 82 & 81.3 \\
    \hline
    $R_{2e2}$ & 47k & 46.7k \\
    \hline
\end{tabular}

\subsection{Tensions de polarisation}
\begin{tabular}{|l|r|r|r|r|}
    \hline
    & Étage 1: théorique & Étage 1: expérimental & Étage 2: théorique & Étage 2: expérimental \\
    \hline
    c & 11.41 & 11.14 & 10.86 & 11.16 \\
    \hline
    b & 1.28 & 1.60 & 5.99 & 4.94 \\
    \hline
    e & 0.59 & 0.96 & 5.40 & 4.47 \\
    \hline
\end{tabular}

\subsection{Tension d’alimentation}
$V_{CC}$ = 12.09V.

\subsection{Amplitudes des tensions}
\begin{tabular}{|l|r|}
    \hline
    entrée & 6.37m \\
    \hline
    sortie du premier étage & 6.30m \\
    \hline
    sortie du second étage & 143.00m \\
    \hline
\end{tabular}

\subsection{Courant total}
Nos deux premiers étages consomment 9.85mA.

\subsection{Fréquences de coupure}

\section{Seconde séance: deux derniers étages}

\subsection{Valeur des résistances}
\begin{tabular}{|l|c|c|}
    \hline
    $R_{3b2}$ & 2200 & 2177 \\
    \hline
    $R_{3c1}$ & 5600 & 5538 \\
    \hline
    $R_{3c1}$ & 5600 & 5585 \\
    \hline
    $R_{3b11}$ & 3900 & 3911 \\
    \hline
    $R_{3b11}$ & 3900 & 3922 \\
    \hline
    $R_{3b11}$ & 3900 & 3929 \\
    \hline
    $R_{3b11}$ & 3900 & 3981 \\
    \hline
    \hline
    $R_{4c}$ & 100 & 99.63 \\
    \hline
    $R_{4e1}$ & 68 & 68.8 \\
    \hline
    $R_{4e2}$ & 1000 & 993.8 \\
    \hline
    $R_{4b1}$ & 330k & 329k \\
    \hline
    $R_{4b2}$ & 680k & 672000 \\
    \hline
\end{tabular}

\subsection{Tensions de polarisation}
Retour sur la polarisation des deux premiers étages, l’impédance de sortie ayant été modifiée, et les tensions d’alimentation aussi:

\begin{tabular}{|l|r|r|r|r|}
    \hline
    & Étage 1: théorique & Étage 1: expérimental & Étage 2: théorique & Étage 2: expérimental \\
    \hline
    c & 11.41 & 11.055 & 10.86 & 10.85 \\
    \hline
    b &  1.28 & 1.577  &  5.99 & 5.982 \\
    \hline
    e &  0.59 & 0.936  &  5.40 & 5.413 \\
    \hline
\end{tabular}

\begin{tabular}{|l|r|r|}
    \hline
    & Étage 3: théorique & Étage 3: expérimental \\
    \hline
    1c           &   6.834 &   6.669 \\
    \hline
    1b           & -12.59m &  -6.62m \\
    \hline
    2c           &   6.834 &   7.067 \\
    \hline
    2b           & -12.59m &  -8.27m \\
    \hline
    1e et 2e     & -647.4m & -0.6167 \\
    \hline
    4c, 3b et 4b &  -4.100 &  -4.984 \\
    \hline
    3e et 4e     &  -4.756 &  -4.698 \\
    \hline
\end{tabular}
\begin{tabular}{|l|r|r|}
    \hline
    & Étage 4: théorique & Étage 4: expérimental \\
    \hline
    c & 11.138 & 11.476 \\
    \hline
    b &  5.137 & 6.181 \\
    \hline
    e &  4.461 & 5.635 \\
    \hline
    s &  4.177 & 5.278 \\
    \hline
\end{tabular}

\subsection{Tension d’alimentation}
$V_{cc}$ = 11.99V.

$V_{ss}$ = -12.05V.

\section{Circuit global}

\subsection{Fréquences de coupure}
$F_{CBF}$ = 98Hz.

$F_{CHF}$ = 190kHz.

\subsection {Impédances}
Entrée: 68000\ohm

Cette impédance est assez choquante, mais en effet, les résistances $R_{1B1}$ et $R_{1B2}$ ne peuvent pas nous permettre d’obtenir la valeur de la résistance d’entrée souhaitée.
