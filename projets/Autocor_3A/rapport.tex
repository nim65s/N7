\documentclass{article}
\usepackage{fontspec}
\usepackage{polyglossia}
\setdefaultlanguage{french}
\usepackage[a4paper,margin=1cm]{geometry}

\usepackage{amsmath}
\usepackage{amssymb}
\usepackage{array}
\usepackage{auto-pst-pdf}
\usepackage{booktabs}
\usepackage{cite}
\usepackage{graphicx}
\usepackage{lmodern}
\usepackage{marvosym}
\usepackage{mathrsfs}
\usepackage{minted}
\usepackage{multicol}
\usepackage{multirow}
\usepackage{paralist}
\usepackage{schemabloc}
\usepackage{siunitx}
\usepackage{soul}
\usepackage{tikz}
\usepackage[european,cuteinductors,siunitx]{circuitikz}
\usepackage{url,hyperref}
\usepackage{verbatim}
\usepackage{xunicode,xltxtra}

\title{\includegraphics{../../../images/inp-enseeiht} \\ ~ \\ ~ \\ ~ \\ ~ \\ Calcul d’une autocorrélation en VHDL}
\author{François Pierron \& Guilhem Saurel}
\date{\oldstylenums{\today}}

\begin{document}

\begin{titlepage}
    \setcounter{page}{0}
    \maketitle
    \vfill
    \tableofcontents
    \thispagestyle{empty}
\end{titlepage}

\section{Introduction}

L’objectif de cette étude est de réaliser une fonction d’autocorrélation sur 16 bits en VHDL, sur des nombres signés, à une fréquence de 200MHz et sur un SPARTAN 6.

\section{Principe}

Au vu du cahier des charges, il est clairement indispensable de paralléliser massivement notre circuit.

* 4x4x4 pour fanout => 4 de retard pour l’autre
* signe géré à la fin (le multiplieur Ca2 est deux fois plus gros)

\section{Resultats}

\section{Conclusion}


\end{document}
