\section{Séance 1: Édition et simulation du VCO, études et statistiques, éditon hiérarchique}
\subsection{Calculs du VCO}
\begin{minted}[linenos]{python}
f0 = Quantity(250, 'kHz')
Df = Quantity(15, 'kHz')
E = Quantity(12, V)
Eb = Quantity(6, V)
Vt = Quantity(0.6, V)
Vcesat = Quantity(0.3, V)
ic = Quantity(10, 'mA')
Rc = (E-Vcesat)/ic
\end{minted}
\verb|Rc = 1.170| \si{\kilo\ohm}

On prend donc la valeur E12 la plus proche:
\begin{minted}[linenos]{python}
Rc = Quantity(1.2, 'kohm')
C = Quantity(150, 'nF')
Dvb = E-Vt
Dvc = E-Vcesat
k = Dvb/Dvc
\end{minted}

$T = 2 R_B C \ln\cfrac{(E_B -V_T)+k(E-V_{CE_{SAT}})}{E_B-V_T} \Rightarrow R_B = \cfrac{1}{2 f_0 C \ln\cfrac{E_B - V_T + k(E -Vcesat)}{E_B-V_T}}$
\begin{minted}[linenos]{python}
Rb = 1 / (2 * f0 * C * log((Eb - Vt + (E - Vcesat)*k)/(Eb - Vt)))
\end{minted}
\verb|Rb = 11.748| \si{\kilo\ohm}

On prend à nouveau la valeur E12 la plus proche:
\begin{minted}[linenos]{python}
Rb = Quantity(12, 'kohm')
\end{minted}
\subsection{Circuit}

\includegraphics[width=\linewidth]{../img/schematic_vco.png}

\subsection{Simulation}

Après une première simulation, on a une fréquence d’oscillation de 238 kHz, ce qui n’entre pas dans les 2\% des 250kHz voulus, mais on pourra jouer sur $E_B$ plus tard avec un potentiomètre, et on pourrait également essayer d’autres valeurs pour les capacités avec les valeurs des résistances $R_B$ appairées.

\includegraphics[width=\linewidth]{../img/oscillations_vco.png}

\subsection{Courbe de la fréquence en fonction de $E_B$}

$K = \cfrac{\Delta f}{\Delta v}$

\includegraphics[width=\linewidth]{../img/variation_freq_vco.png}

\begin{minted}[linenos]{python}
Df2 = 79802
Dv2 = 2.9115
K = Df2/Dv2
\end{minted}
\verb|K = 27409.239223767814| \si{\Hz/\V}
\subsection{Analyse de Monte Carlo}

Pour chaque paramètre à étudier, on fait 500 simulations pour une distribution Gaussiene du paramètre, et on affiche un histograme des fréquences obtenues.
On devrait donc pouvoir comparer les 4 histogrammes obtenus pour voir rapidement quel paramètre a le plus d’influence.

Cependant, en regardant les équations, on s’attend à ce que les paramètres les plus influents sur la période soient R et C.

* On commence par regarder l’influence des paramètres R et C:

\includegraphics[width=\linewidth]{../img/montecarlo_R_C.png}


* Ensuite on regarde l’influence des paramètres du transistor, en commençant par $\beta$:

\includegraphics[width=\linewidth]{../img/montecarlo_beta.png}


On remarque que dans certains cas, la tension observée diverge, et spice nous en alerte.

* Puis $C_{jc}$:

\includegraphics[width=\linewidth]{../img/montecarlo_Cjc.png}


* Et enfin $C_{je}$

\includegraphics[width=\linewidth]{../img/montecarlo_Cje.png}


Comme on s’y attendait, les paramètres les plus influents sont de loins R et C.
En effet, une variation de ceux-ci modifie fait varier la fréquence d’une centaine de \si{\kilo\hertz}, alors que pour les autres c’est plutôt de l’ordre de 5 \si{\kilo\hertz}.
\subsection{Hiérarchisation du VCO}

En suivant le polycopié «Orcad-Pspice Hierarchie et sous-circuit», nous avons changé notre schématique en une véritable librairie, et la schématique peut maintenant se comporter comme n’importe quel autre composant dans une autre schématique.

\includegraphics[width=\linewidth]{../img/schematic_low_hierarchy.png}


\includegraphics[width=\linewidth]{../img/schematic_high_hierarchy.png}


\includegraphics[width=\linewidth]{../img/schematic_vco_final.png}


On effectue enfine une simulation temporelle pour vérifier que tout continue de fonctionner:

\includegraphics[width=\linewidth]{../img/simu_temporelle_bloc_hierarchique.png}

\subsection{Signal modulant BF}

On change enfin BC par un générateur sinusoïdal:

\includegraphics[width=\linewidth]{../img/variation_freq_sinus_en_entree.png}

Cette simulation montre bien que la fréquence du VCO varie avec la tension qu’on lui applique.

\section{Séance 2: Implémentation sur plaquette Labdec et mesures du VCO}
\subsection{Signal d’entrée de test}

Pour tester notre VCO, nous connectons son entrée à un potentiomètre suivi d’un amplificateur opérationnel suiveur, pour pouvoir faire varier la fréquence du VCO en modifant sa tension d’entrée via le potentiomètre.

\includegraphics[width=\linewidth]{../img/schematic_test_vco.png}

\section{Séance 3: Simulation et mesures de l’amplificateur audio}

%TODO: Schéma p.54, bode théorique p.53

$G_0 = \cfrac{\Delta f}{K} = \cfrac{R_2}{R_1}\cfrac{R_5}{R_3}$

D’après le cahier des charges, $R_1 = 50 \si{\kilo\ohm}$ donc on prendra $R_1 = 47 \si{\kilo\ohm}$.

On a choisi à la séance précédente $R_5 = 10 \si{\kilo\ohm}$, donc pour simplifier les calculs, on prendra $\cfrac{R_5}{R_3} = 1 \Rightarrow R_3 = 10 \si{\kilo\ohm}$.

Il nous manque donc plus que $R_2$, qu’on détermine grace à $\Delta f = 15 kHz$ et $K = 26.5 kHz$.

Donc $R_2 = 27 \si{\kilo\ohm}$. De plus, $f_2 = 6 kHz \Rightarrow C_2 = \cfrac{1}{f_2 2 \pi 27000} = 1 nF$.

($f_2 = \cfrac{1}{2\pi R_4 C_4}$ \& $f_1 = \cfrac{1}{2\pi C_4(R_3 + R_4}$ ) $\Rightarrow$ ($C_4 = 17 nF \simeq 18nF$ \& $R_4 = 1473 Ohm \simeq 1.5 \si{\kilo\ohm}$).

$f_3 = f_4 = f_{CBF}\sqrt{\sqrt{2}-1} = 32 Hz$.

$f_3 = \cfrac{1}{2\pi C_3R_3} \Rightarrow C_3 = \cfrac{1}{32 \cdot 2 \pi \cdot 10000} = 500nF \simeq 470 nF$.

$f_4 = \cfrac{1}{2\pi C_1 (R_gR_1)} \simeq \cfrac{1}{2\pi R_1C_1} \Rightarrow C_1 = \cfrac{1}{32 \cdot 2 \pi \cdot 47000} = 100nF$.
\section{Séance 4: Simulation et mesures du circuit DEL}
\subsection{I − Circuits commutés}

$V_F=1.8V$

$I_0=100mA$

$R_{SC}\simeq 2Ohm$

$V_{CE_{sat}}=R_{SC}I_0=0.2V$

$E=R_6I_0+V_F+V_{CE_{sat}}\Rightarrow R6=\cfrac{E-V_F-V_{CE_{sat}}}{I_0}=100Ohm$
\begin{minted}[linenos]{python}
E = Quantity(12, V)
Vf = Quantity(1.8, V)
I0 = Quantity(0.1, A)
Rsc = Quantity(2, ohm)
Vcesat = Rsc * I0
\end{minted}
\verb|Vcesat = 200| \si{\milli\volt}
\begin{minted}[linenos]{python}
R6 = ( E - Vf - Vcesat) / I0
\end{minted}
\verb|R6 = 100| \si\ohm

$P_6=\cfrac{R_6I_0^2}{2}=0.5W$
\begin{minted}[linenos]{python}
P6 = R6 * I0**2 / 2
\end{minted}
\verb|P6 = 500| \si{\milli\watt}

$V_{E5}\simeq E-V_T$

$I_{B6}\simeq I_{C5}=\cfrac{V_{E5}}{R_{E5}}$

$R_{B6}=\cfrac{V_{E5}-V_{BE6}}{I_{B6}}=\cfrac{E-2V_T}{I_{B6}}$

On cherche a obtenir $I_{B6}$ supérieur à $\cfrac{I_0}{\beta}$ ; donc $R_{E5}\simeq R_{B6}\simeq 10\si{\kilo\ohm}$ font l’affaire.
\begin{minted}[linenos]{python}
Re5 = Quantity(10, 'kohm')
Ve5 = E-Vt
\end{minted}
\verb|Re5 = 10| \si{\kilo\ohm}
\begin{minted}[linenos]{python}
Rb6 = Re5 * (Ve5 - Vt) / Ve5
\end{minted}
\verb|Rb6 = 9.474| \si{\kilo\ohm}
\begin{minted}[linenos]{python}
Rb6 = Quantity(10, 'kohm')
\end{minted}
\begin{minted}[linenos]{python}
Ic5 = Ve5 / Re5
\end{minted}
\verb|Ic5 = 1.14| \si{\milli\ampere}

$R_{C5}\simeq\cfrac{V_T-V_{CE5}}{I_{C5}}$

Si on prend $V_{CE5} = 0.3V$, faible mais supérieur à $V_{CE_{sat}}=0.2V$, on obtient $R_{C5}=263 Ohm \simeq 270 Ohm$
\begin{minted}[linenos]{python}
Vce5 = Quantity(0.3, V)
Rc5 = (Vt - Vce5) / Ic5
\end{minted}
\verb|Rc5 = 263.158| \si\ohm
\subsection{III − Réduction des temps de montée}
\subsubsection{1 − Réduction de $t_{ri}$}

$C_6 > \cfrac{\beta}{2\pi f_TR_{B6}}$
\begin{minted}[linenos]{python}
fT = Quantity(250, 'kHz')
C6 = 100 / (Rb6 * 2 * pi * fT)
\end{minted}
\verb|C6 = 6.366| \si{\nano\farad}

On prend alors 100 \si{\nano\farad}.
\section{Séance 5: Routage de l’émetteur, dépôt du typon}
Ce routage s’est déroulé sans problème, voici à quoi ressemble notre typon:

\includegraphics[width=\linewidth]{../img/routage.png}

\section{Séance 6: Soudage de la carte de l’émetteur − mesures de l’ḿetteur complet}
Cette séance mérite assez peu de commentaires techniques, si ce n’est «Ça marche».

Par contre, on peut ajouter qu’on est particulièrement content que ça marche aussi bien.
