\documentclass{article}
\usepackage{fontspec}
\usepackage{polyglossia}
\setdefaultlanguage{french}
\usepackage[a4paper,margin=2cm]{geometry}

\usepackage{amsmath}
\usepackage{array}
\usepackage{auto-pst-pdf}
\usepackage{booktabs}
\usepackage{cite}
\usepackage{graphicx}
\usepackage{lmodern}
\usepackage{marvosym}
\usepackage{mathrsfs}
\usepackage{minted}
\usepackage{multicol}
\usepackage{multirow}
\usepackage{paralist}
\usepackage{schemabloc}
\usepackage{siunitx}
\usepackage{soul}
\usepackage{tikz}
\usepackage[european,cuteinductors,siunitx]{circuitikz}
\usepackage{url,hyperref}
\usepackage{verbatim}
\usepackage{xunicode,xltxtra}

\title{\includegraphics{../../../../images/inp-enseeiht} \\ ~ \\ ~ \\ ~ \\ ~ \\ Projet Analogique}
\author{Vincent Pétillat, Guilhem Saurel}
\date{\oldstylenums{\today}}

\begin{document}

\begin{titlepage}
    \setcounter{page}{0}
    \maketitle
    \thispagestyle{empty}
\end{titlepage}

\tableofcontents

\section*{Préambule}

Ceci est notre rapport du Projet Analogique dans le cadre du cours de M. Bernal
en deuxième année d’Électronique et traitement du signal à l’INP-ENSEEIHT.

Il est en grande partie généré à partir des quelques notes que nous avons pris
pendant les différentes séances, et inclue un maximum de captures d’écran du logiciel
Orcad-Pspice que nous avons utilisé pour ce projet, ainsi que du code Python
qui nous a servi pour nos différents calculs.

Ce code Python est formatté grace au plugin LaTeX «minted», et vous pourrez donc
facilement le repérer avec sa fonte monospaced, sa coloration syntaxique et
la numérotation des lignes. Les blocs de code sont parfois suivis de la sortie
standart de ce code, dans un environnement verbatim (monospaced noir).

\subsection*{Remerciements}
Un grand merci à nos encadrants, très présents, qui nous ont permis d’avancer rapidement:
\begin{itemize}
    \item M. A.;
    \item M. B.;
\end{itemize}

On remercie également les logiciels qui nous ont permis d’avancer rapidemend et proprement:
\begin{itemize}
    \item IPython, et son fameux Notebook;
    \item XeTex;
\end{itemize}

\include{1-genere}

\end{document}
