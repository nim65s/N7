\documentclass{article}
\usepackage{fontspec}
\usepackage{polyglossia}
\setdefaultlanguage{french}
\usepackage[a4paper,margin=1cm]{geometry}

\usepackage{amsmath}
\usepackage{amssymb}
\usepackage{array}
\usepackage{auto-pst-pdf}
\usepackage{booktabs}
\usepackage{cite}
\usepackage{graphicx}
\usepackage{lmodern}
\usepackage{marvosym}
\usepackage{mathrsfs}
\usepackage{minted}
\usepackage{multicol}
\usepackage{multirow}
\usepackage{paralist}
\usepackage{schemabloc}
\usepackage{siunitx}
\usepackage{soul}
\usepackage{tikz}
\usepackage[european,cuteinductors,siunitx]{circuitikz}
\usepackage{url,hyperref}
\usepackage{verbatim}
\usepackage{xunicode,xltxtra}

\title{\includegraphics{../../../images/inp-enseeiht} \\ ~ \\ ~ \\ ~ \\ ~ \\ Initiation cadence layout XL}
\author{François Pierron \& Guilhem Saurel}
\date{\oldstylenums{\today}}

\begin{document}

\begin{titlepage}
    \setcounter{page}{0}
    \maketitle
    \vfill
    \tableofcontents
    \thispagestyle{empty}
\end{titlepage}

\section{Design and Simulation of the electrical schematic of the operational amplifier (Cadence Composer schematic editor and Spectre)}

\begin{table}[|l|c|}
    \hline
    Electrical Characteristics & simulation Results \\
    \hline
    Power Consumption & \\
    \hline
    Open Loop Gain & \\
    \hline
    Bandwidth & \\
    \hline
    Offset & \\
    \hline
    Input Voltage nois (Power spectrum Density) & \\
    \hline
    Phase Margin & \\
    \hline
    Gain Margin & \\
    \hline
    Slew Rate & \\
    \hline
\end{table}


\section{Layout of the schematic using the technology rule sets provided by the fondery AMS for this 0.35$\mu$m CMOS process}


\section{Simulations post-layout}


\end{document}
