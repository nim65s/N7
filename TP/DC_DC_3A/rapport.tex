\documentclass{article}
\usepackage{fontspec}
\usepackage{polyglossia}
\setdefaultlanguage{french}
\usepackage[a4paper,margin=1cm]{geometry}

\usepackage{amsmath}
\usepackage{amssymb}
\usepackage{array}
\usepackage{auto-pst-pdf}
\usepackage{booktabs}
\usepackage{cite}
\usepackage{graphicx}
\usepackage{lmodern}
\usepackage{marvosym}
\usepackage{mathrsfs}
\usepackage{minted}
\usepackage{multicol}
\usepackage{multirow}
\usepackage{paralist}
\usepackage{schemabloc}
\usepackage{siunitx}
\usepackage{soul}
\usepackage{tikz}
\usepackage[european,cuteinductors,siunitx]{circuitikz}
\usepackage{url,hyperref}
\usepackage{verbatim}
\usepackage{xunicode,xltxtra}

\title{\includegraphics{../../../images/inp-enseeiht} \\ ~ \\ ~ \\ ~ \\ ~ \\ TP Convertisseur DC-DC}
\author{François Pierron \& Guilhem Saurel}
\date{\oldstylenums{\today}}

\begin{document}

\begin{titlepage}
    \setcounter{page}{0}
    \maketitle
    \vfill
    \tableofcontents
    \thispagestyle{empty}
\end{titlepage}


\section{Open Loop Study}

\paragraph{Observe the currents I(L), I(D1) et I(S1). Explain the behavior of currents according to the gate signal. In what mode are you ??}

%\includegraphics[width=\linewidth]{vs.png}
%\includegraphics[width=\linewidth]{courants.png}

\paragraph{Observe the control signals: Vrampe, Verreur, Vgrille and the resulting switched voltage Vk.}

%\includegraphics[width=\linewidth]{commandes.png}

\subparagraph{What is the value of the duty-cycle ?}

~

En theorie, cette valeur vaut $\cfrac{V_S}{V_{bat}} = \cfrac{5}{12} = 0.416$.

En pratique, on a $0.427$.

\subparagraph{Why do not you get exactly 5V for the voltage Vs by imposing D=5V/12V?}

~

La diode en inverse impose un abaissement de l’etat bas du rapport cyclique de $V_D$. $V_S$ a donc une tension approximativement de $5-\cfrac{V_D}{2}$.


%\includegraphics[width=\linewidth]{vk.png}

\subparagraph{Provide the expression of Vs=f(D,Vbat,Vd) avec Vd: knee voltage of the diode.}

~

$V_S = D V_{bat} - (1 - D)V_d$

\subparagraph{Deduce the voltage value to be imposed on Verreur to get Vs=5V.}

~

$V_{erreur}=2.22$V

\subparagraph{Check with a simulation.}

~

Après simulation, il semble que 2.235 corresponde mieux.

%\includegraphics[width=\linewidth]{vs223.png}

\paragraph{Measure the value of the current ripple and the output voltage ripple and compare with the theorical values.}

~

$\Delta I_L = \cfrac{D(1-D)V_{bat}}{Lf_{dec}} = 0.0899$ A
%\includegraphics[width=\linewidth]{ripple_il.png}

En pratique, en est à 97mV.

$\Delta V_S = \cfrac{\Delta I_L}{8Cf_{dec}} = 0.0112$ V
%\includegraphics[width=\linewidth]{ripple_vs.png}

En pratique, on est à 12mV.

Donc dans les deux cas, la pratique est assez proche de la thorie.

\paragraph{By varying the load, check that the boundary separating modes is obtained for $R_{ch_{limite}} = \cfrac{2Lf}{1-D}$ with f: switching frequency (100kHz)}

~

La resistance limite entre les modes continu et discontinu est telle que le courant moyen de sortie est egal à la moitie du courant de ripple:
$R_{ch_{lim}} = \cfrac{V_S}{\cfrac{\Delta I_L}{2}} = 111\Omega$.

\subparagraph{Compare theory and simulation for both voltage: Vbat=12V and Vbat=20V}

~

On trouve que $V_D \simeq 0.9$V, donc, en thorie, on a $D = \cfrac{V_S-V_D}{V_{bat}-V_D}$.
Ceci conduit à des valeurs theoriques des resistances de $121 \Omega$ à 12V et de $91 \Omega$ à 20V ; et nous obtenons bien $117\Omega$ dans le premier cas, et $103\Omega$ dans le second.


\paragraph{In discontinuous mode, for Vbat=12V, Rcharge=300$\Omega$ and D=0.46, measure Vs (in steady-state) and check the expression
    $D=\cfrac{V_s}{V_{bat}}\cdot\sqrt{\cfrac{2Lf}{R_{charge}\left(1-\cfrac{V_s}{V_{bat}}\right)}}$.}

~

%\includegraphics[width=\linewidth]{discontinuous.png}

$V_S=7.38$V, donc $D=0.46$


\paragraph{Observe (in simulation) the value of the overall efficiency of the circuit close to the nominal point (5V - 1A).}

~

La moyenne du courant consomm sur l’interrupteur est de 465mA
La moyenne du courant consomme par la resistance de chareg est de 996mA pour une tension 4.98 V, ce qui fait à peu près 4.96W, d’où un rendement de 0.90.



\end{document}
