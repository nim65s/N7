\documentclass{article}
\usepackage{fontspec}
\usepackage{polyglossia}
\setdefaultlanguage{french}
\usepackage[a4paper,margin=1cm]{geometry}

\usepackage{amsmath}
\usepackage{amssymb}
\usepackage{array}
\usepackage{auto-pst-pdf}
\usepackage{booktabs}
\usepackage{cite}
\usepackage{graphicx}
\usepackage{lmodern}
\usepackage{marvosym}
\usepackage{mathrsfs}
\usepackage{minted}
\usepackage{multicol}
\usepackage{multirow}
\usepackage{paralist}
\usepackage{schemabloc}
\usepackage{siunitx}
\usepackage{soul}
\usepackage{tikz}
\usepackage[european,cuteinductors,siunitx]{circuitikz}
\usepackage{url,hyperref}
\usepackage{verbatim}
\usepackage{xunicode,xltxtra}

\title{\includegraphics{../../../images/inp-enseeiht} \\ ~ \\ ~ \\ ~ \\ ~ \\ TP Simulation de fautes}
\author{François Pierron \& Guilhem Saurel}
\date{\oldstylenums{\today}}

\begin{document}

\begin{titlepage}
    \setcounter{page}{0}
    \maketitle
	\tableofcontents
    \thispagestyle{empty}
\end{titlepage}

www.aime-toulouse.fr/CAO/COURSCAO/tp_fautes/

000, 111 \& 011 => couverture de 88.9\%, manque sa1 pour I{2,4}.{A,B}

011, 111, 010, 001, 100

1 00 00 00
1 10 00 00
1 11 11 00
1 11 11 11
0 10 10 10
1 01 01 01
0 01 01 01

1 0 2 3 2 3 3 3 3 3 3 2 3 3 3 3 3 3 3 3 3 
96.2
manquait les RST


0 0 0 0 1 0 0 0 1 1 1 0 91.5

\end{document}

