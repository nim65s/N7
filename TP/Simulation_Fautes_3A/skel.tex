\documentclass[11pt]{article}
\usepackage{fontspec}
\usepackage{polyglossia}
\setdefaultlanguage{french}
\usepackage[a4paper,margin=3cm]{geometry}

\usepackage{amsmath}
\usepackage{amssymb}
\usepackage{array}
\usepackage{auto-pst-pdf}
\usepackage{booktabs}
\usepackage{cite}
\usepackage{graphicx}
\usepackage{lmodern}
\usepackage{marvosym}
\usepackage{mathrsfs}
\usepackage{minted}
\usepackage{multicol}
\usepackage{multirow}
\usepackage{paralist}
\usepackage{schemabloc}
\usepackage{siunitx}
\usepackage{soul}
\usepackage{tikz}
\usepackage[european,cuteinductors,siunitx]{circuitikz}
\usepackage{url,hyperref}
\usepackage{verbatim}
\usepackage{xunicode,xltxtra}

\title{\includegraphics{../../../images/inp-enseeiht} \\ ~ \\ ~ \\ ~ \\ ~ \\ TP Simulation de fautes}
\author{François Pierron \& Guilhem Saurel}
\date{\oldstylenums{15 novembre 2013}}

\begin{document}

\begin{titlepage}
    \setcounter{page}{0}
    \maketitle
    ~ \\ ~ \\ ~ \\ ~ \\ ~ \\ ~ \\
    ~ \\ ~ \\ ~ \\ ~ \\ ~ \\ ~ \\
    ~ \\ ~ \\ ~ \\ ~ \\ ~ \\
    \tableofcontents
    \thispagestyle{empty}
\end{titlepage}

\section{Étude du 1er circuit: full adder}

Si l’on applique la séquence \verb|<000, 011, 111>|, on remarque que chaque ligne est placée à 0 et à 1, mais que la couverture n’est que de 88.9\%, puisqu’il manque sa1 pour I2.A, I2.B, I4.A et I4.B.

On peut obtenir une couverture de 100\% avec la séquence:
\inputminted[linenos]{verilog}{full_adder.v}

\section{Test de 3 additionneurs cascadés: adder3}

Pour détecter toutes les pannes, il nous faut utiliser la séquence:
\inputminted[linenos]{verilog}{adder3.v}


\section{Test d’un compteur synchrone: compt4}

Voici une séquence de test qui détermine 96.2\% des pannes :
\inputminted[linenos]{verilog}{compt4.v}

\section{Test d’un automate au niveau graphe d’états: MEF}

Pour passer une fois par chacun des états internes, on peut appliquer à l’entrée la séquence:
\inputminted[linenos]{verilog}{MEF.v}

(en commentaires, il y a l’état courant)

Cette séquence a une couverture de 91.5\%.

\section{Simulation de la mémoire}

Obtenir une couverture de 100\% est relativement simple ; mais l’obtenir en un minimum de lignes ne l’est pas.
Notre score est de 48:
\inputminted[linenos]{verilog}{ram100.v}

\end{document}

