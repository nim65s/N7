\documentclass[11pt;a4paper]{article}
\usepackage[top=2cm, bottom=2cm, left=2cm, right=2cm]{geometry}
\usepackage[utf8]{inputenc}
\usepackage{fontenc}
\usepackage[french]{babel}
\usepackage{lmodern}
\usepackage{amsmath}
\usepackage{amssymb}
\usepackage{mathrsfs}
\usepackage{multirow}
\usepackage{url,hyperref}
\usepackage{circuitikz}

\begin{document}

$(F(p))^* = F(p^*)$
Hurwitz : coef réels à coefs dont la partie réele est positive
Déphasage minimal : pas de zéro à partie réelle positive
PR : p réel entraine F(p) réel et Re(p) >=0 entraine Re(F(p)) >=0
PR => synthétisable par une immittance d'un dipole passif
PR CNS : 
1) F(p) est homorphe dans le demi plan droit strictement.
2) Les pôles de F(p) sur l’axe imaginaire sont simples et les résidus de F(p) en ces pôles sont des réels positifs.
3) Si l’infini est pôle, il est d’ordre 1 et le résidu correspondant est un réel positif :

4) U(w) est positive ou nulle quelle que soit la valeur de w

Forme : $H(p) = K_\infty p + \cfrac{K_0}{p} + \sum_i \cfrac{2 K_i p}{p^2+\omega_i^2}, K > 0$

NON PR :
1) Elle content des pôles ou des zéros de multiplicité supérieure à l’unité à l’infini ou sur l’axe imaginaire.
2) Elle a des coefficients négatifs.


III

H(p) a nécessairment un pôle ou un zéro à l'origine et un pôle ou un zéro à l'infini
Formes de Foster :
\begin{circuitikz} \draw
    (0,1) to[L=$A_\infty$] (3,1)
          to[C=$\cfrac{1}{A_0}$] (5,1) -- (5,0)
          to[C, l_=$\cfrac{1}{A_i}$] (7,0) -- (7,2)
          to[L=$\cfrac{A_i}{\omega_i^2}$] (5,2) -- (5,1)
          ;
\end{circuitikz}
\begin{circuitikz} \draw
    (2,4) to[L=$\cfrac{1}{A_0}$] (2,0)
    (4,4) to[C=$A_\infty$] (4,0)
    (6,4) to[C=$\cfrac{A_i}{\omega_i^2}$] (6,2)
          to[L=$\cfrac{1}{A_i}$] (6,0)
    (0,0) -- (6,0)
    (0,4) -- (6,4)
    ;
\end{circuitikz}

Formes de Cauer : (première si pole à l'infini
$Z(p) = L_1 p + \cfrac{1}{C_1 p + \cfrac{1}{L_2 p + \cfrac{1}{C_2 p + \cfrac{1}{\dots}}}}$
\begin{circuitikz} \draw
    (0,2) to[L=$L_1$] (2,2) to[C=$C_1$] (2,0)
    (2,2) to[L=$L_i$] (4,2) to[C=$C_i$] (4,0) -- (0,0)
    ;
\end{circuitikz}

% TODO 3p17

$Z(p) = \cfrac{1}{L_1 p} + \cfrac{1}{\cfrac{1}{C_1 p} + \cfrac{1}{\cfrac{1}{L_2 p} + \cfrac{1}{\cfrac{1}{C_2 p }+ \cfrac{1}{\dots}}}}$
\begin{circuitikz} \draw
    (0,2) to[C=$C_1$] (2,2) to[L=$L_1$] (2,0)
    (2,2) to[C=$C_i$] (4,2) to[L=$L_i$] (4,0) -- (0,0)
    ;
\end{circuitikz}

HPLP : $\Omega = \cfrac{-\omega_c}{\omega}$
BPLP : $\Omega = \cfrac{\omega}{\omega_2-\omega_1} +\cfrac{\omega_1 \omega_2}{p(\omega_2 - \omega_1)}$
L : $ (l) L\Delta \omega / \omega_0^2 \parallel (c) 1/\Delta \omega$
C : $ (l) 1/C \Delta \omega -- (c) C \Delta \omega / \omega_0^2$

Butterworth : $g_r = 2 \sin ( \cfrac{2r-1}{2n} \pi )$
Tchebycheff : $T_n(x) = cos(n arcos(x)) si |x| < 1 ou ch(n argch(x))$

\end{document}
