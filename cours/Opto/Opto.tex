%% This file was auto-generated by IPython.
%% Conversion from the original notebook file:
%% Opto.ipynb
%%
\documentclass[11pt,english]{article}

\usepackage{fontspec}
\usepackage{polyglossia}
\setdefaultlanguage{french}
\usepackage[a4paper,margin=1cm]{geometry}

\usepackage{amsmath}
\usepackage{amssymb}
\usepackage{array}
\usepackage{auto-pst-pdf}
\usepackage{booktabs}
\usepackage{cite}
\usepackage{graphicx}
\usepackage{lmodern}
\usepackage{marvosym}
\usepackage{mathrsfs}
\usepackage{minted}
\usepackage{multicol}
\usepackage{multirow}
\usepackage{paralist}
\usepackage{schemabloc}
\usepackage{siunitx}
\usepackage{soul}
\usepackage{tikz}
\usepackage[european,cuteinductors,siunitx]{circuitikz}
\usepackage{url,hyperref}
\usepackage{verbatim}
\usepackage{xunicode,xltxtra}

\begin{document}

\section{Introduction}
Optique: Champs électromagnétiques où
$10 \si{\nano\metre} < \lambda < 1 \si{\nano\metre} $.

Historique: Découvert en 1887 par Hertz et expliqué par Einstein en
1917.

DEL~en 60, LASERS en 58 à 66, fibres entre 66 et 77.

Domaines: des télécoms aux LANs, l'éclairage, écrans (petits \& grands),
capteurs (gaz, mécanique, image), médical (chirurgie,
épilation,\ldots{}), électronique (DEL)

\section{Photométrie et rayonnement du corps noir}
\subsection{Introduction}
Photons -\textgreater{} dualité onde-crépuscule ; on jonglera entre les
deux.

\begin{itemize}
\item
  Onde : $\lambda$, $f = \cfrac{c}{\lambda}$, $v_g = \cfrac{c}{n}$ ($n$
  indice de réfraction)
\item
  Particule : $E = h\nu$, où
  $h = 6.6\cdot 10^{-34} \si{\watt\per\square\second}$, {[}Ws{]} =
  {[}J{]}, mais $1 eV = 1.6\cdot 10^{-19}\si{\joule}$ ; donc
  $E = \cfrac{1.24}{\lambda}$
\end{itemize}

Afin de prendre en compte la sensibilité aux photons de l'œil humain :
$1\si{\lumen} = 1.6\cdot 10^{-3}\si{\watt} à \lambda = 550 \si{\nano\metre}$
(centre de la bande passante de l'œil)

\subsection{Grandeurs photométriques}
\subsubsection{Le flux lumineux F}
Def : valeur instantanée d'un débit de rayonnement, en $\si{\watt}$ (ou
en photons par seconde)

\subsection{L’angle solide $\Omega$}
$A = \Omega\cdot r^2 \Rightarrow d\Omega=\cfrac{dA}{R^2}=2\pi\sin{\theta}d\theta$

\subsubsection{La luminance L}
Def: Flux par unité de surface apparente de la source et par unité
d'angle solide.

$L = \cfrac{dF}{\cos\theta dA_sd\Omega}$, en
$\si{\watt\per\square\metre\per\steradian}$

Quand $\theta$ influe peu sur la luminance, on dit que la source rayonne
selon la \textbf{loi de Lambert} :

$I = \cfrac{dF}{d\Omega} = \iint{L\cos\theta dA_s} \Rightarrow I = \cos\theta\iint{LdA_s} = I_0\cos\theta$

La courbe $I(\theta)$ est appelée \textbf{indicatrice} de la source.

\subsubsection{L’émittance M}
Def : le flux émis par unité de surface:
$M = \cfrac{dF}{dA_s} = \int{L\cos\theta d\Omega}$, en
$\si{\watt\per\square\metre}$

\textbf{loi de Lambert} : $M = L\int\cos\theta d\theta$, et
$F = \int{MdA_s} = LA_s\int\cos\theta d\theta = 2\pi LA_s\int\cos\theta\sin\theta d\theta=\pi I_0 \Rightarrow I_0 = \cfrac{F}{\pi}$

\subsubsection{L’éclairement}
Récepteur de lumière: on définit $A_R$ et $\theta_R$

Un point du récepteur reçoit un flux élémentaire $dF$

Def : L'éclairement E en $\si{\watt\per\square\metre}$ est le rapport du
flux par la surface du récepteur : $dE = \cfrac{d^2F}{dA_R}$

Donc $dF = L\cos\theta_sdA_sd\Omega_R$ où
$d\Omega_R=\cfrac{dA_r\cos\theta_R}{d^2} \Rightarrow dE = L\cfrac{dA_s\cos\theta_s\cos\theta_R}{d^2}$

Cas d'une source ponctuelle:
$E = \cfrac{\cos\theta_r}{d^2}\int{LdA_s\cos\theta_s}=I\cfrac{\cos\theta_r}{d^2}$,
donc l'éclairement évolue en $\cfrac{1}{d^2}$

\subsection{Unités Spécifiques et ordres de grandeur}
Trois types d'unités: * Énergétiques * Photoniques * Liées à l'œil

En gros, on passe de l'énergétique aux photoniques en passant des
$\si{\watt}$ au $\si{\per\second}$, mais les photoniques dépendent de
$\lambda$.

\subsection{Le rayonnement du Corps Noir}
Def: Un corps noir est un radiateur qui absorbe parfaitement les
rayonnements incidents et qui émet un spectre continu de radiations.

\subsubsection{Loi de Planck}
Planck décrit l'émittance d'un corps noir:
$\cfrac{dM(\lambda,T)}{d\lambda}$=$\cfrac{2\pi hc^2\lambda^{-5}}{\exp{\cfrac{hc}{\lambda k_BT}}-1}$

\subsubsection{Loi de Wien}
$\lambda_M = \cfrac{2898}{T}$

$\cfrac{dM(\lambda,T)}{d\lambda} = 1.286\cdot 10^{-5}\cdot T^5$ en
$\si{\watt\cubic\metre}$.

\subsubsection{Loi de Stefan-Boltzmann}
$M_T = \sigma T^4$ en $\si{\watt\per\square\metre}$, où
$\sigma = \cfrac{2\pi^5k_B^4}{15c^2h^3}$

Donc le corps humain, qui fait en gros 37 degrés et
$2\si{\square\metre}$,
$M_T = 500 \si{\watt\per\square\metre} \Rightarrow F = 1 \si{\kilo\watt}$.

Pour le filament de tyngstène,
$T = 5800\si{\kelvin} \Rightarrow M_T = 348\cdot 10^4 \si{\watt\per\square\metre} \Rightarrow F = 38.8\si\watt$

\subsection{La Fibre Optique}
\subsection{Introduction}
\subsubsection{Pourquoi de la Fibre Optique}
\begin{itemize}
\itemsep1pt\parskip0pt\parsep0pt
\item
  augmentation de la quantité des données échangées,
\item
  distances très grandes,
\item
  perturbations EM
\end{itemize}

La FO:

\begin{itemize}
\itemsep1pt\parskip0pt\parsep0pt
\item
  bande passante quasi illimitée
\item
  atténuation très faible (0.2dB/km =\textgreater{} 250km sans répéteur)
\item
  immunité aux perturbations EM
\item
  très légère
\end{itemize}

Par contre, la mise en œuvre est très cher.

\subsection{Structure d’une FO}
\subsubsection{Loi de Snell-Descartes}
$\theta_{lim} = \arcsin\cfrac{n_2}{n1}$

\subsubsection{Guidage optique}
Structure:

\begin{itemize}
\itemsep1pt\parskip0pt\parsep0pt
\item
  Cœur: 3μm à 65.5μm
\item
  Gaine: 125μm
\item
  Gaine de protection: plastique, kevlar ou carbone
\end{itemize}

\subsection{Différents types de FO}
Fibres multimodes: l'équation de propagation a plusieurs solutions.

\begin{itemize}
\itemsep1pt\parskip0pt\parsep0pt
\item
  Saut d'indice $n_c$ constant
\item
  Gradient d'indice
  $n_c(r)=n_c\sqrt{1 - 2\Delta\left(\cfrac{r}{a}\right)^g}$ où
  $\Delta = \cfrac{n_c^2-n_g^2}{n_c^2}\simeq\cfrac{n_c-n_g}{n_c} < 1\%$
\end{itemize}

Fibres monomodes: Saut d'indice, mais diamètre de cœur faible.

\subsection{Quelques grandeurs caractéristiques des FO}
\subsubsection{Ouverture numérique}
Pour une FO~à saut d'indice,

$ON = \sin(\theta_{ext_{lim}}) \Rightarrow \theta_{1_{max}} = \arcsin\cfrac{n_g}{n_c} \Rightarrow n_{ext}\sin\theta_{ext}=n_c\sin\theta_i=n_c\cos\theta_1\Rightarrow ON = \cfrac{n_c}{n_{ext}}\cos\theta_{1_{max}}=\cfrac{n_c}{n_{ext}}\cos\left(\arcsin\cfrac{n_g}{n_c}\right)=\cfrac{n_c}{n_{ext}}\sqrt{1-\cfrac{n_g^2}{n_c^2}}\Rightarrow ON = \cfrac{\sqrt{n_c^2-n_g^2}}{n_{ext}}$

Donc $n_c(r)\sin\theta(r)=n_c(r+dr)\sin\theta(r+dr)$ constant. La limite
de guidage est atteinte en r=a:
$n_c(a)\sin\theta(a)=n_g\Rightarrow ON_{FOI}=\cfrac{\sqrt{n_c(r)^2-n_g^2}}{n_{ext}}$

Couplage avec une DEL:

En première approximation, c'est une source ponctuelle Lambertienne.

$dF=LA_s\cos\theta d\Omega$

$F_{inj}=\int_0^{\theta_{max}}dF=2\pi LA_s\int_0^{\theta_{max}}\sin\theta\cos\theta d\theta=\pi LA_s\sin^2\theta_{max}=\pi LA_sON^2$

$F_{tot}=\pi LA_s\Rightarrow\cfrac{F_{inj}}{F_{tot}}=ON^2$

A.N.~:$n_c=1.53,n_g=1.5\Rightarrow ON=0.3$

On n'injecte donc dans ce cas là que 9\% de la puissance\ldots{}

\subsection{La fréquence normalisée}
C'est un paramètre issu de la résolution des équations de propagation:

$V=\cfrac{2\pi}{\lambda}a\sqrt{n_c^2-n_g^2}$

Il est sans unités, donc aucun rapport avec une fréquence\ldots{}

\begin{itemize}
\itemsep1pt\parskip0pt\parsep0pt
\item
  La condition de propagation monomode est : $V < 2.41$, donc une fibre
  est monomode pour une plage limitée de longueurs d'ondes.
\item
  $V \gg 1\Rightarrow N_{modes}\simeq\cfrac{V^2}{2}$
\end{itemize}

\subsection{Phénomènes limitants dans les fibres}
\subsubsection{L’atténuation}
$E(t,z)=E_0\exp^{j\omega t}\exp^{j\beta z - \alpha^\prime z}$

On préfère utiliser A en dB/km, tq:
$Az=10\log\left(\cfrac{P(z)}{P(0)}\right)$

Donc pour FO~de longeur L, pour des flux en dBm,
$F_{sortie}=F_{entrée}-AL$

Rappels:

\begin{itemize}
\itemsep1pt\parskip0pt\parsep0pt
\item
  1mW: 0dBm
\item
  1W: 30dBm
\item
  on peut ajouter des dB et des dBm
\end{itemize}

\subsubsection{Autres types de pertes}
\begin{itemize}
\itemsep1pt\parskip0pt\parsep0pt
\item
  Pertes par épissurage (raccordemenetde 2 FO par fusion)

  \begin{itemize}
  \itemsep1pt\parskip0pt\parsep0pt
  \item
    Fibres non identiques
  \item
    Espacement
  \item
    Excentrement
  \item
    Désalignement
  \end{itemize}
\item
  Pertes par courbure au refroidissement
\item
  Pertes par couplage à la source

  \begin{itemize}
  \itemsep1pt\parskip0pt\parsep0pt
  \item
    rien : 10\%
  \item
    fibre lentillée: \textgreater{}60\%
  \item
    couplage à une lentille: \textgreater{}75\%
  \item
    montage confocal à deux lentilles: \textgreater{}85\%
  \end{itemize}
\end{itemize}

A.N. :

\begin{itemize}
\itemsep1pt\parskip0pt\parsep0pt
\item
  laser 1, 10mW, 850nm (old, mais bien fichu)
\item
  laser 2, 10mW, 1550nm (récent, mais cher)
\end{itemize}

$P_{inj}=10dBm$

\begin{tabular}{|l|c|c|c|}
\hline
 & 1km & 10km & 100 km \\
\hline
Laser 1 & 7dBm & 20dBm & 190 dBm \\
\hline
Laser 2 & & 8dBm & 10dBm \\
\hline
\end{tabular}

\subsubsection{Dispersion dans les FO}
Def: C'est un élargissement temporel des impulsions transmises

C'est un phénomène statistique, pour le quantifier, on utilise la notion
d'écart-type $\sigma_{\Delta t}$

slides 23, 24

Il existe deux types de dispersion dans les FO

\begin{itemize}
\itemsep1pt\parskip0pt\parsep0pt
\item
  dispersion modale
\item
  dispersion chromatique
\end{itemize}

Dispersion modale:

Dans une FO multimode, chaque mode porte une partie de la puissance du
signal transmi et le temps de propagation dans la FO~est propre à chaque
mode, avec
$\Delta t_{max} = \cfrac{l_{max} - l_{min}}{v} = \cfrac{n_c}{c}\left( \cfrac{L_{FO}}{\sin\theta_{lim}}-L_{FO} \right)$,
or
$\sin\theta_{lim} = \cfrac{n_g}{n_c} \Rightarrow \Delta t_{max} = \cfrac{n_cL_{FO}}{c}\cfrac{n_c-n_g}{n_g}$

Rappel :
$ON = \cfrac{\sqrt{n_c^2-n_g^2}}{n_0}\simeq\cfrac{2n_g(n_c-n_g)}{n_0}$\ldots{}
$n_0 = 1\Rightarrow\delta t_{max} = \cfrac{L (ON)^2}{2cn_c}$

Ecart-type: En première approximation,
$\sigma_{\Delta t} = \cfrac{\Delta t_{max}}{2} = \cfrac{L_{FO} (ON)^2}{4cn_c}$

Pour une FO à saut d'indice, on défini le terme de dispersion modale
$D_m$: $\sigma_{\Delta t}=D_m L_{FO}$.

$D_m$ est un {[}s/m{]} ou en {[}ps/km{]}

$D_{m_{FSI}} = \cfrac{(ON)^2}{4cn_c}$

\begin{itemize}
\itemsep1pt\parskip0pt\parsep0pt
\item
  Longueur d'équilibre : les modes d'ordre le plus élevés sont
  progressivement absorbés par la gaine, donc l'impact de la dispersion
  diminue.
\end{itemize}

On considère que

$L_{FO} < L_{eq} \Rightarrow \sigma_{\Delta t}=D_m L_{FO}$

$L_{FO} > L_{eq} \Rightarrow \sigma_{\Delta t}=D_m \sqrt{L_{eq}L_{FO}}$

L'ordre de grandeur de $L_{eq}$ est de 1km pour une FSI

\begin{itemize}
\itemsep1pt\parskip0pt\parsep0pt
\item
  Cas d'une FO à gradient d'indice (FGI)
\end{itemize}

$n_c(n) = n_c\sqrt{1-2\Delta\left(\cfrac{r}{q}\right)^2} \Rightarrow D_{m_{FGI}} = \cfrac{n_c(0)\Delta^2}{8c}$,
avec $\Delta =\cfrac{nc-ng}{n_c}$

Donc $\cfrac{D_{m_{FGI}}}{D_{m_{FSI}}}=\cfrac{n_c^2-n_g^2}{4}$, avec
$n_c^2-n_g^2 << 1$

AN: FSI: $n_c=1,53;n_g=1.5\Rightarrow ON=0.3$

FGI: $n_c=1,53;n_g=1.5\Rightarrow ON=0.02$

$D_{m_{FSI}} = 49 \si{\nano\second\per\kilo\metre}$

$D_{m_{FGI}} = 255 \si{\pico\second\per\kilo\metre}$

\begin{itemize}
\itemsep1pt\parskip0pt\parsep0pt
\item
  La dispersion chromatique
\end{itemize}

Deux choses:

\begin{itemize}
\itemsep1pt\parskip0pt\parsep0pt
\item
  Les sources de lumière ne sont pas totalement monochromatiques
\item
  L'indice de réfraction d'un milieu dépend de la longueur d'onde
\end{itemize}

$n_c(\lambda) \Rightarrow v_g=\cfrac{c}{n}$ est fonction de $\lambda$

$\Delta_{v_g} = \cfrac{dv_g}{d\lambda}\Delta\lambda$

$\Delta T = -T\cfrac{\Delta v_g}{v_g} = - \cfrac{L_{FO}\Delta v_g}{v_g v_g}$

$\Delta T = -\cfrac{lambda}{c}…${[}\ldots{}{]}

$\Delta T = D_\lambda \Delta_\lambda L_{FO}$

$D_\lambda$ est le terme de dispersion chromatique:
$D_\lambda=-\cfrac{\lambda}{c}\cfrac{d^2x}{d\lambda^2}$ en
{[}$\si{\pico\second\per\nano\metre\per\kilo\metre}${]}

$\sigma_{\Delta t_lambda} = \vert D_\lambda\vert \sigma_{\Delta_\lambda}L_{FO}$

slide 25

on a un écart temporel lié à une distribution de la puissance dans le
domaine spectral ($\sigma_{\Delta_\lambda}$)

$D_\lambda = 0$ à 1330nm

$D_\lambda = 17 \si{\pico\second\per\nano\metre\per\kilo\metre}$ à
1550nm

$D_\lambda = -80 \si{\pico\second\per\nano\metre\per\kilo\metre}$ à
850nm

*Bande passant d'une FO : cas où les deux types de dispersion sont
présents: FO multimodes

$\sigma_{\Delta t_{tot}} = \sqrt{\sigma_{\Delta t_m}^2 + \sigma_{\Delta t_\lambda}^2}$

Par définition, $BP = \cfrac{0.35}{\sigma_{\Delta t_{tot}}}$

AN: FGI multimode : $n_c=1,53;n_g=1.5$

$D_{m_{FGI}} = 255 \si{\pico\second\per\kilo\metre}$

Laser $\lambda=850 \si{\nano\metre}$,
$\sigma_{\Delta_\lambda} = 10 \si{\nano\metre}$

$D_\lambda = -80\si{\pico\second\per\nano\metre\per\kilo\metre}$

$L_{FO} = 1\si{\kilo\metre}$

$\sigma_{\Delta t_m} = 255\si{\pico\second}$

$\sigma_{\Delta t_\lambda} = 800\si{\pico\second}$

$\sigma_{\Delta t_{tot}} = 840\si{\pico\second}$

$BP = 1.2$ GHz pour 1km

AN2: FGI multimode : $n_c=1,53;n_g=1.5$

$D_{m_{FGI}} = 255 ps/km$

Laser $\lambda=1550$nm, $\sigma_{\Delta_\lambda} = 0.1$nm

$D_\lambda = 17$ ps/nm/km

$L_{FO} = 1$ km

$\sigma_{\Delta t_m} = 255$ ps

$\sigma_{\Delta t_\lambda} = 1.7$ ps

$\sigma_{\Delta t_{tot}} \simeq 255$ ps : la dispersion chromatique est
devenue négligeable.

$BP = 1.2$ GHz pour 1km

AN3: FO~monomode:

$D_m = 0$

Laser $\lambda=1550$nm, $\sigma_{\Delta_\lambda} = 0.1$nm

$BP = 600$ GHz/km (6Ghz pour 100 km)

slide 26, 27

/!~Connaitre la slide 27 /!\\

\subsection{Les FO plastiques}
\subsubsection{Structure}
\begin{itemize}
\itemsep1pt\parskip0pt\parsep0pt
\item
  Fibres multimodes à saut d'indice
\item
  Pas de gaine
\item
  Les matériaux

  \begin{itemize}
  \itemsep1pt\parskip0pt\parsep0pt
  \item
    polystyrène (pas cher, mais pas terrible)
  \item
    polycarbonate (tient la chaleur =\textgreater{} automobiles)
  \item
    PMMA (top) : polyméthylmétacrylate
  \end{itemize}
\end{itemize}

\subsubsection{Avantages et inconvénients}
\begin{itemize}
\itemsep1pt\parskip0pt\parsep0pt
\item
  BP plus faible (au moins 10 fois)
\item
  Atténuation: 50 à 100 dB/km pour le PMMA
\item
  diamètre à cœur: 1mm
\item
  ouverture numérique élevée =\textgreater{} facilité d'injection de la
  lumière
\item
  coût plus faible
\item
  possibilité d'applications médicales jettables
\end{itemize}

\section{La photodétection}
Il existe deux classe de photodétecteurs:

\begin{itemize}
\itemsep1pt\parskip0pt\parsep0pt
\item
  Quantique
\item
  Thermique (lent et peu sensible, mais utilisés pour la puissance, donc
  pas en télécoms)
\end{itemize}

Parmis les détecteurs quantiques :

\begin{itemize}
\itemsep1pt\parskip0pt\parsep0pt
\item
  photoconductances
\item
  phototransistors (MOS dont la grille est pilotée par le flux lumineux)
\item
  capteurs matriciels CCD ou MOS
\item
  photodiodes (99\% des applications hors photo)
\end{itemize}

\subsection{L’Absorption des photons dans un semi-conducteur}
Quand un photo arrive avec une énergie supérieure à $h_\nu$, paf un
électron passe de la couche de valence à la couche de conduction.

Il existe une longeur d'onde maximale pour laquelle il n'y a plus
d'absorption: $\lambda_g = \cfrac{hc}{E_g} = \cfrac{1.24}{E_g}$, avec
$\lambda$ en $\mu$m et $E_g$ en eV

Le flux de photons dans un semiconducteur :
$F(\lambda) = F_0\exp^{-\alpha x}$, où $\alpha$ est l'absorption en
$\si{\per\metre}$

slide 31

\subsection{Principe de la photodiode}
\subsubsection{Polarisatino d’une PD}
Pour toute diode,
$I = I_s\left(\exp^{\cfrac{V}{nU_T}} - 1\right) - I_{ph}$

Ce $I_{ph}$ décale le diagrame $I=f(V)$ vers le bas, ce qui fait que
pour des tensions positives, le courant est négatif, donc la PD fournit
de l'énergie au circuit ; et pour des tensions négatives, le courant est
négatif aussi, donc la PD~fournit de l'énergie au circuit. On se place
généralement dans ce troisième cadrant pour la photodétection.

\subsubsection{Modèle petit signal}
\begin{circuitikz} \draw
  (0,0) node {A}
    to[diode] (0,2) node {K}
  (8,0) -- (2,0) to [current source=$I_{ph}$] 
  (2,2) -- (6,2) to [R=$R_s$] (8,2)
  (4,0) to [C=$C_j$] (4,2)
  (6,0) to [R=$R_{sh}$] (6,2)
  ;
\end{circuitikz}

$R_{sh} > \si{\mega\ohm}$

$R_s < \si\ohm$

$C_j(V)=\cfrac{C_j(V=0)}{\sqrt{1+\cfrac{V}{V_{PN}}}}$

$V_{PN}$ est le potentiel de jonction: 0.75V dans le silicium

\subsubsection{Photodiode PIN}
slide 29(32)

Pour diminuer $C_j$, on rajoute à la jonction PN une couche faiblement
dopée («Intrinsèque») qui va étendre la ZCE: plus la ZCE~est importante,
plus la capa de jonction est faible.

\begin{itemize}
\item
  Structure

  \begin{itemize}
  \itemsep1pt\parskip0pt\parsep0pt
  \item
    La zone frontale doit àtre la plus fine possible
  \item
    La ZCE est le lieu de collecte des photons
  \item
    La zone arrière voit très peu de photons
  \end{itemize}
\item
  Bande-Passante
\end{itemize}

Elle est liée:

Au temps de transit des porteurs:

$ t_{tr} = \cfrac{d}{v_n} \simeq 10\si{\pico\second} $

La constante de temps liée à la capa de jonction : 

\begin{circuitikz} \draw
  (0,0) node[ground]{} to[R=$R_{load}$] (0,2) to[diode] (0,4);
\end{circuitikz}
$\rightarrow f_c=\cfrac{1}{2\pi R_{load}C_j} $

Des deux phénomènes, le plus contraignant impose la Bande Passante

\begin{itemize}
\itemsep1pt\parskip0pt\parsep0pt
\item
  Sensibilité d'une diode PIN:
\end{itemize}

$R_\lambda = \cfrac{I_{ph}}{F_{recu}}$ en A/W (R pour Responsivity)

$F_{recu}$ = nombre de photons reçus par seconde × $h_\nu$

$\nu_q=\cfrac{\text{nombre de paires electron-trou par seconde}}{\text{nombre de photons par seconde}}$

nombre de paires electron-trou par seconde =
$\cfrac{I_{ph}}{q} \Rightarrow R_\lambda=\cfrac{q}{h_\nu}\nu_q=\cfrac{\lambda}{1.24}\nu_q$
pour $\lambda$ en $\si{\micro\metre}$

\subsubsection{Bruit dans les PD}
La notion de bruit regroupe deux aspects:

\begin{itemize}
\itemsep1pt\parskip0pt\parsep0pt
\item
  Les perturbations extérieures au système: en général, ce sont des
  signaux déterministes qui dépendent de l'environnement.
\item
  Les perturbations propres au système: phénomènes aléatoires qui
  peuvent être traités par la physique quantique ou la physique
  statistique.
\end{itemize}

Dans le cas du bruit en électronique, deux sources:

\begin{itemize}
\item
  Le bruit thermique (ou de Johnson) associé aux résistances
  $\Rightarrow$ variation aléatoire du courant qui traverse la
  résistance ou de la tension à ses bornes.
\item
  Le bruit de grenaille (ou de Shottky) associé aux jonctions PN
  $\Rightarrow$ variation du courant qui traverse la jonction
\item
  La puissance de bruit
\end{itemize}

La puissance du bruit est infinie (intégrale d'un bruit blanc sur tout
le spectre), donc on défini toujours une puissance de bruit sur une
bande passante donnée.

\begin{itemize}
\itemsep1pt\parskip0pt\parsep0pt
\item
  La densité spectrale de puissance ($S_B$)
\end{itemize}

C'est la puissance du bruit sur 1\si{\hertz}

$P_B = \int_0^{BP}S_Bdf$, avec $S_B$ en \si{\watt\per\hertz}

Bruit thermique et grenailles sont des bruits blancs, donc
$P_B=S_B\times BP$

\begin{itemize}
\itemsep1pt\parskip0pt\parsep0pt
\item
  Densité spectrale de courant et de tension
\end{itemize}

$S_B=S_iR_{load}$, $S_i$ en \si{\square\ampere\per\hertz}

$S_B=\cfrac{S_v}{R_{load}}, $S\_v\$ en \si{\square\volt\per\hertz}

Pour le bruit thermique, $S_i=\cfrac{4k_BT}{R}$ et $S_v=4k_BTR$

\begin{circuitikz} \draw
	(0,0) to[R] (0,2) -- (2,2) to [current source] (2,0) -- (0,0)
	(4,1) to[R] (5,1) to[voltage source](7,1);
\end{circuitikz}

Pour le bruit de grenaille, $S_i=2qI$

\begin{circuitikz} \draw
	(0,0) to[R] (0,2) -- (2,2) to [current source] (2,0) -- (0,0);
\end{circuitikz}

\begin{itemize}
\itemsep1pt\parskip0pt\parsep0pt
\item
  Courant ou tension de bruit
\end{itemize}

On définit $i_d=\sqrt{S_i}$ en \si{\ampere\per\hertz\tothe{0.5}} et
$v_d=\sqrt{S_v}$ en \si{\volt\per\hertz\tothe{0.5}}\$

Ce sont des valeursefficaces calculées sur une bande de 1\si\hertz.

Rem: lois de Kirschoff

\begin{circuitikz} \draw
	(0,2) to[short,i=$i_{b1}$] (2,2) to [short,i=$i_{bT}$] (4,2)
	(2,0) to[short,i=$i_{b2}$] (2,2);
\end{circuitikz}
$\Rightarrow i_{b_T} = \sqrt{i_{b_1}^2+i_{b_2}^2}$

TODO vb1 \textbar{} vb2 − vbt \textbar{} =\textgreater{}
$v_{b_T} = \sqrt{v_{b_1}^2+v_{b_2}^2}$

\begin{itemize}
\itemsep1pt\parskip0pt\parsep0pt
\item
  Modèle en bruit de la photodiode
\end{itemize}

\begin{circuitikz} \draw
	(0,0) to[current source=$i_d$] (0,2) -- (2,2) to [C=$C_j$] (2,0) -- (0,0);
\end{circuitikz}

Le courant que parcours la PD~est la somme de deux termes:

\begin{itemize}
\itemsep1pt\parskip0pt\parsep0pt
\item
  Le courant $I_{ph}$ photodétecté
\item
  $I_{obs}$, courant d'obscurité
\end{itemize}

$i_d=\sqrt{S_i} = \sqrt{2q(I_{ph}+I_{obs})}$

\subsubsection{Puissance équivalente de bruit}
La PD transforme la puissance optique en courant électrique
=\textgreater{} un bruit électrique est équivalent un bruit optique:

$PEB=\cfrac{i_{BT}}{R_\lambda}$ en W/$\sqrt{\text{Hz}}$

$P_{B_{\text{optique}}} = PEB\sqrt{BP_{det}}$ avec
$BP_{det} = \cfrac{\pi}{2}BP_{\text{systeme}}$ si le système est du
premier ordre.

Exemple:

TODO diode en inverse, VCE en haut, patte inverseuse d'un ampli diff en
bas C\_f // R\_f en contre réaction, v\_out en sortie

$V_{out} = \cfrac{R_f}{1+R_fC_fp}i_{out}$

$f_c=\cfrac{1}{2\pi R_fC_f}$

On considère l'ampli comme idéal (pas de bruit)

$S_i=2q < i_{ph} > + 2qI_{obs} + \cfrac{4kT}{R_f}$

$i_{bt} = \sqrt{S_i}=\sqrt{2q(I_{ph}+I_{obs})+\cfrac{4kT}{R_f}}$ TODO:
I\_ph =\textgreater{}

$PEB=\cfrac{\sqrt{2q(I_{ph}+I_{obs})+\cfrac{4kT}{R_f}}}{R_\lambda}$
TODO: I\_ph =\textgreater{} 

\subsubsection{Photodiode à avalanche (APD)}
On impose un champ E interne dans la ZCE, les porteurs sont fortement
accélerés, ils libèrent d'autre paires electron-trou par transfert
d'énergie cinétique, qui sont elles-même fortement accélérées.

slide 30(33)

\begin{itemize}
\itemsep1pt\parskip0pt\parsep0pt
\item
  Structure:
\end{itemize}

Le champ intense est réalisé par une zone à très for dopage.

On réalise une zone large de faible dopage qui permet de collecter les
photons.

\begin{itemize}
\itemsep1pt\parskip0pt\parsep0pt
\item
  Gain d'une APD
\end{itemize}

$G=\cfrac{I_{APD}}{I_{ph}}=\cfrac{1}{1-\left(\cfrac{V_{APD}-R_sI_{APD}}{V_B}\right)^m}$,
$V_B$ est la tension de claquage, $m \in [3,6]$

\begin{itemize}
\itemsep1pt\parskip0pt\parsep0pt
\item
  Bande passante
\end{itemize}

La limitation de la bande passante est dûe au temps de transit des
porteurs dans la zone de génération plus le temps de multiplication

$t_d=\cfrac{d_g}{v_n} + \cfrac{d_g}{v_p} + t_{\text{mult}}$

NB: $t_{d_{APD}} \gg T_{d_{PIN}}$

\begin{itemize}
\itemsep1pt\parskip0pt\parsep0pt
\item
  Domaines d'utilisation
\end{itemize}

\begin{tabular}{|c|c|c|}
\hline
& PIN & APD \\
\hline
BP & + & - \\
\hline
sensibilité & - & + \\
\hline
mise en œuvre & ++ & - \\
\hline
\end{tabular}

Principalement des applications de puissance optique constante de très
faible intensité (spatial)

AN: $F_{\text{recu}} = 0.1\mu W$, APD: $R_\lambda = 280$ A/W,

le système de détection a un bruit de détection propre dans l'obscurité
$S_{i_{obs}} = 3.3\cdot 10^{-23}$ A²/Hz

=\textgreater{} \textless{}$I_{ph}$\textgreater{} = 28 muA
=\textgreater{} $S_{i_g} = 8.96\cdot10^{-24}$A²/Hz

BP = 1GHz =\textgreater{} PEB = $2.3\cdot10^{-14}$W/$\sqrt{Hz}$
=\textgreater{} $P_{b_{opt}} = 7.3\cdot10^{-4} \mu$ W

Soit $RSB = \cfrac{0.1}{7.3\cdot10^{-4}}\simeq 100$

Si la PD est une PIN, avec $R_\lambda = 0.5$, $RSB\simeq0.15$

\section{Diodes electroluminescentes}
\subsection{Effet photoélectriques}
\subsubsection{Rappel paire électron-trou}
Dans tout matériau semi-conducteur, il existe des électrons faiblements
liés à la maille cristalline. Sous l'effet d'un apport d'énergie
(agitation thermique, champ EM\ldots{}), ces électrons vont s'arracher
de la maille cristalline, passant ainsi de la bande de valence à la
bande de conduction. Ils laissent localement une lacune de charge
électrique assimilable à une particule de charge positive: le trou. On a
alors une paire électron-trou.

L'énergie nécessaire à un éloctron pour passer de la valence à la
conduction est $E_g=E_c-E_v$. La valeur de $E_g$ est propre à chaque
semi-conducteur (slide 28)

\subsubsection{Effets photoélectriques}
Il existe trois types :

\begin{itemize}
\itemsep1pt\parskip0pt\parsep0pt
\item
  Absorption: un photon d'énergie $h_\nu > E_g$ arrive sur un électron
  dans la couche de valence;
\item
  Émission spontanée: une paire électron-trou se recombine, et libère
  une énergie $E_g$ sous forme d'un photon;
\item
  Émission stimulée: un photon incident déclenche la recombinaison. Ce
  photon n'est pas absorbé, mais un nouveau photon est généré, dont
  l'énergie est strictement égale à celle du photon incident. Ils ont la
  même longueur d'onde et sont en phase.
\end{itemize}

Rmq: Toutes les recombinaisons ne sont pas radiatives

\begin{itemize}
\itemsep1pt\parskip0pt\parsep0pt
\item
  Il existe des «pièges», c'est à dire des niveaux d'énergie dans la
  bande interdite dûs aux impuretés du cristal
\item
  Effet Auger: Transfert de l'énergie entre 3 particules (deux électrons
  et un trou ou deux trous et un électron) sans création de photons
\end{itemize}

Rmq: Il existe de l'absorption non photoélectrique

\subsubsection{La jonction P-N}
Il faut deux conditions pour émettre beaucoup de photons:

\begin{itemize}
\itemsep1pt\parskip0pt\parsep0pt
\item
  probabilité élevée de recombinaisons radiatives (choix du matériau)
\item
  densité élevée de porteurs libres (jonction PN)
\end{itemize}

La joction PN polarisée en direct entraîne une très forte densité de
recombinaisons au niveau de la ZCE

\subsubsection{Les matériaux}
Ils permettent de choisir la longueur d'onde: l'énergie des photons
générés est
$h_\nu = h \cfrac{c}{\lambda} = E_g \Rightarrow \lambda = \cfrac{1.24}{E_g}$
avec $\lambda$ en $\mu$m et $E_g$ en eV.

Il y a un lien direct entre l'énergie de gab du SC et la longueur d'onde
émise.

Transition directe ou indirecte

Les matériaux à transition directe ont une probabilité élevée de
recombinaison radiative, contrairement à ceux à transition indirecte.

Le choix du matériau fait intervenir la longueur d'onde émmise et le
type de transition. Les matériaux alliages de composants des colonnes
III et IV du cableau de Mendeleiev sont utilisé pour la réalisation
d'émetteurs

\subsubsection{Largeur spectrale d’émission}
La probabilité d'émission à partir d'un porteur de quantité de mouvement
non nulle est élevée.

L'évolution da l adensité de porteurs en fonction de leur énergie évolue
en $\exp^{\cfrac{-E}{k_BT}}$

Mais la densité de niveau d'énergie disponibles évolue en $\sqrt{E-E_g}$

TODO:~plot une expo négaive et une sqrt, ainsi qu'une courbe qui passe
sous les deux autres, avec une largeur à mi-hauteur de
$\Delta E = 1.8k_B$

Or
$\lambda = \cfrac{1.24}{E}\Rightarrow\cfrac{\Delta\lambda}{\lambda}=\cfrac{\Delta E}{E}\Rightarrow\Delta\lambda=\cfrac{\Delta E}{E_g}\lambda=\cfrac{\Delta E}{1.24}\lambda^2$

Donc la largeur spectrale à mi-hauteur
$\Delta\lambda=\cfrac{1.8k_BT}{1.24}\lambda^2$

\subsection{La DEL}
\subsubsection{Structure d’une DEL}
C'est une diode dont la métallisation supérieur est aménagée afin de
permettre la sortie des photons. La couche de semi-conducteur supérieure
est fine pour limiter l'absorption. On rajoute des couches isolantes
sous la métallisation pour guider les porteurs sous la fenêtre
d'émission (slides 29 et 30)

\subsubsection{Rendements}
\begin{itemize}
\itemsep1pt\parskip0pt\parsep0pt
\item
  Rendement d'injection:
  $\eta_{inj}=\cfrac{\text{Nombre de porteurs injectés dans la zone P par seconde}}{\text{Nombre total de porteurs injectés par seconde}} \simeq 1$
\item
  Rendement radiatif:
  $\eta_{rad}=\cfrac{\text{Nombre de recombinaisons rad/d}}{\text{Nombre de porteurs injectés dans P/s}}\simeq 1$
\item
  Rendement associé à l'angle critique: d'après Snell-Descartes,
  $\theta_c=\arcsin\cfrac{n_0}{n_{sc}}\simeq 17°$ pour l'AsGa
  ($n_{sc}=3.5$).
  $F_{out} = \int_0^{\theta_c}I_{interne}d\Omega=\int_0^{\theta_c}I_{interne}2\pi\sin\theta d\theta=2\pi I_{interne}\int_0^{\theta_c}\sin\theta d\theta=2\pi I_{interne}(1-\cos\theta)\Rightarrow\eta_{critique}=\cfrac{F_{out}}{F_{total}} = \cfrac{1-\cos\theta_c}{2}$.
  Pour augmenter ce rendement, on enrobe la DEL~d'un diélectrique
  transparent d'indice $n_d>n_0$. On atteint ainsi les 5\%
\item
  Rendement d'absorption: $\eta_{abs}=\exp^{-\alpha x}$, où $\alpha$ est
  le coefficient d'asorption (slide 31) et x l'épaisseur de la couche
  supérieure du semi-conducteur.
\item
  Pertes de Fresnel: il y a de la réflexion à l'interface, et le
  coefficient le réflexion est
  $R=\left(\cfrac{n_{sc}-n_0}{n_{sc}+n_0}\right)^2$, et
  $\eta_{fresnel}=1-R\simeq 70\%$ pour de l'AsGa
\end{itemize}

Rendement total :
$\eta_{tot}=\eta_{inj}\eta_{rad}\eta_{crit}\eta_{abs}\eta_{fresnel}$

\subsubsection{Puissance optique émise}
\begin{itemize}
\itemsep1pt\parskip0pt\parsep0pt
\item
  Calcul de la puissance :
\end{itemize}

Nombre d'électrons injectés = $\cfrac{I_{DEL}}{q}$

Flux émis:
$F_e = N_{photons/s}=h_\nu\Rightarrow F_e=eta_{tot}\cfrac{I_{DEL}}{q}h_\nu$,
or $h_\nu=E_g$ et
$\cfrac{E_g}{q}=\cfrac{1.24}{\lambda}\Rightarrow F_e=\eta_{total}\cfrac{1.24}{\lambda}I_{DEL}$

\begin{itemize}
\itemsep1pt\parskip0pt\parsep0pt
\item
  Effets thermiques: aux courants élevés, les rendements d'injection et
  radiatifs chutent et la puissance émise plafonne.
\end{itemize}


\end{document}
