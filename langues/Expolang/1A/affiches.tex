\documentclass[11pt;a4paper]{article}
\usepackage[utf8]{inputenc}
\usepackage{fontenc}
\usepackage[french]{babel}
\usepackage[top=3cm, bottom=3cm, left=3cm, right=3cm]{geometry}
\usepackage{graphicx}

\newcommand{\theu}[0]{\textsuperscript{th}}
\newcommand{\st}[0]{\textsuperscript{st}}
\newcommand{\nd}[0]{\textsuperscript{nd}}
\newcommand{\rd}[0]{\textsuperscript{rd}}

\pagestyle{empty}

\begin{document}

\section*{Queen Elisabeth II}

\subsection*{Elizabeth II}

Elizabeth II (Elizabeth Alexandra Mary, born April 21\st 1926) is the constitutional monarch of sixteen independent sovereign states known as the Commonwealth realms: the United Kingdom, Canada, Australia, New Zealand, Jamaica, Barbados, the Bahamas, Grenada, Papua New Guinea, the Solomon Islands, Tuvalu, Saint Lucia, Saint Vincent and the Grenadines, Belize, Antigua and Barbuda, and Saint Kitts and Nevis. In addition, as Head of the Commonwealth, she is the figurehead of the 54-member Commonwealth of Nations and, as the British monarch, she is the Supreme Governor of the Church of England.

Elizabeth was educated privately at home. Her father, George VI, became in 1936 King of the United Kingdom and the British Dominions and Emperor of India. She began to undertake public duties during the Second World War, in which she served in the Auxiliary Territorial Service. After the war and Indian independence George VI's title of Emperor of India was abandoned, and the evolution of the British Empire into the Commonwealth accelerated. In 1947, Elizabeth made the first of many tours around the Commonwealth, and married Prince Philip, Duke of Edinburgh. They have four children: Charles, Anne, Andrew, and Edward.

In 1949, George VI became the first Head of the Commonwealth, a "symbol of the free association of its independent member nations". On his death in 1952, Elizabeth became Head of the Commonwealth, and queen of seven independent Commonwealth countries: the United Kingdom, Canada, Australia, New Zealand, South Africa, Pakistan, and Ceylon. Her coronation in 1953 was the first to be televised. During her reign, which at 59 years is one of the longest for a British monarch, she became queen of 25 other countries within the Commonwealth as they gained independence. Between 1956 and 1992, half of her realms, including South Africa, Pakistan, and Ceylon (renamed Sri Lanka), became republics.

In 1992, which Elizabeth termed her annus horribilis ("horrible year"), two of her sons separated from their wives, her daughter divorced, and a severe fire destroyed part of Windsor Castle. Revelations on the state of her eldest son Charles's marriage continued, and he divorced in 1996. The following year, her former daughter--in--law Diana, Princess of Wales, died in a car crash in Paris. The media criticised the royal family for remaining in seclusion in the days before Diana's funeral, but Elizabeth's personal popularity rebounded once she had appeared in public and has since remained high. Her Silver and Golden Jubilees were celebrated in 1977 and 2002; planning for her Diamond Jubilee in 2012 is underway.

As a granddaughter of the monarch in the male line, Elizabeth's full style at birth was Her Royal Highness Princess Elizabeth of York. She was third in the line of succession to the throne, behind her uncle, Edward, Prince of Wales, and her father. Although her birth generated public interest, she was not expected to become queen, as the Prince of Wales was still young, and it was widely assumed that he would marry and have children of his own. In 1936, when her grandfather, the King, died and her uncle Edward succeeded, she became second in line to the throne after her father. Later that year, Edward abdicated after his proposed marriage to divorced socialite Wallis Simpson provoked a constitutional crisis. Elizabeth's father became king, and she became heiress presumptive, with the style Her Royal Highness The Princess Elizabeth. 

\subsection*{Arms}

\newpage

\subsection*{Anecdotes during Second World War}

During the Second world War, the suggestion by senior politician Lord Hailsham that the two princesses should be evacuated to Canada was rejected by Elizabeth's mother; she declared, ``The children won't go without me. I won't leave without the King. And the King will never leave.''Windsor, the princesses staged pantomimes at Christmas in aid of the Queen's Wool Fund, which purchased yarn to knit into military garments. It was from Windsor in 1940 that the 14-year-old Elizabeth made her first radio broadcast during the BBC's Children's Hour, addressing other children who had been evacuated from the cities. She stated:

``We are trying to do all we can to help our gallant sailors, soldiers and airmen, and we are trying, too, to bear our share of the danger and sadness of war. We know, every one of us, that in the end all will be well.''

In 1943, at the age of 16, Elizabeth undertook her first solo public appearance on a visit to the Grenadier Guards, of which she had been appointed Colonel-in-Chief the previous year. In February 1945, she joined the Women's Auxiliary Territorial Service, as an honorary Second Subaltern with the service number of 230873. She trained as a driver and mechanic, drove a military truck, and was promoted to honorary Junior Commander five months later. She is the last surviving head of state who served in uniform during the Second World War.

Two years later, the princess made her first overseas tour, when she accompanied her parents to Southern Africa. During the tour, in a broadcast to the British Commonwealth on her 21\st birthday, she pledged: ``I declare before you all that my whole life, whether it be long or short, shall be devoted to your service and the service of our great imperial family to which we all belong.''

\subsection*{Finances}

Elizabeth's personal fortune has been the subject of speculation for many years. Forbes magazine estimated her net worth at around 450 million in 2010, but official Buckingham Palace statements in 1993 called estimates of \pounds 100 million ``grossly overstated'', and Jock Colville estimated her wealth at \pounds 2 million in 1971 (the equivalent of about \pounds 21 million today). The Royal Collection, which includes artworks and the Crown Jewels, is not owned by the Queen personally and is held in trust, as are the occupied palaces in the United Kingdom such as Buckingham Palace and Windsor Castle, and the Duchy of Lancaster, a property portfolio valued at \pounds 48 million in 2010. As was so with many of her predecessors, Elizabeth is reported to dislike Buckingham Palace as a residence, and prefers Windsor Castle. Sandringham House and Balmoral Castle are privately owned by the Queen. Income from the British Crown Estate -- with holdings of \pounds 6.6 billion in 2010 -- is transferred to the British treasury in return for Civil List payments. Both the Crown Estate and the Crown Land of Canada -- comprising 89\% of Canada's area -- are owned by the Sovereign in trust for the nation, and cannot be sold or owned by Elizabeth in a private capacity. 

\newpage

\section*{Nicolas Sarkozy}

\subsection*{Biography}

Nicolas Sarkozy (born Nicolas Paul Stéphane Sarközy de Nagy-Bocsa; January 28\theu 1955) is the 23\rd and current President of the French Republic and ex officio Co-Prince of Andorra. He assumed the office on May 16\theu 2007 after defeating the Socialist Party candidate Ségolène Royal 10 days earlier.

Before his presidency, he was leader of the Union for a Popular Movement (UMP). Under Jacques Chirac's presidency he served as Minister of the Interior in Jean-Pierre Raffarin's (UMP) first two governments (from May 2002 to March 2004), then was appointed Minister of Finances in Raffarin's last government (March 2004 to May 2005) and again Minister of the Interior in Dominique de Villepin's government (2005-2007).

Sarkozy was also president of the General council of the Hauts-de-Seine department from 2004 to 2007 and mayor of Neuilly-sur-Seine, one of the wealthiest communes of France from 1983 to 2002. He was Minister of the Budget in the government of Édouard Balladur (RPR, predecessor of the UMP) during François Mitterrand's last term.

Sarkozy is known for wanting to revitalize the French economy. He has pledged to revive the work ethic, promote new initiatives and fight intolerance. In foreign affairs he has promised a strengthening of the entente cordiale with the United Kingdom and closer cooperation with the United States. He married singer-songwriter Carla Bruni on February 2\nd 2008 at the Élysée Palace in Paris.

\subsection*{Studies}

Sarkozy was enrolled in the Lycée Chaptal, a state-funded public middle and high school in Paris's 8\theu arrondissement, where he failed his sixième. His family then sent him to the Cours Saint-Louis de Monceau, a private Catholic school in the 17\theu arrondissement, where he was reportedly a mediocre student, but where he nonetheless obtained his baccalauréat B in 1973. He enrolled at the Université Paris X Nanterre, where he graduated with an MA in Private law with the tiniest mark acceptable. Later he gratuated a DEA degree in Business law. Paris X Nanterre had been the starting place for the May '68 student movement and was still a stronghold of leftist students. Described as a quiet student, Sarkozy soon joined the right-wing student organization, in which he was very active. He completed his military service as a part time Air Force cleaner. After graduating, he entered the Institut d'Études Politiques de Paris, better known as Sciences Po, (1979-1981) but failed to graduate due to an insufficient command of the English language. After passing the bar, he became a lawyer specializing in business and family law, and was one of Silvio Berlusconi's top French advocates.

\newpage

\subsection*{Marriages}
\subsubsection*{Marie-Dominique Culioli}

Sarkozy married his first wife, Marie-Dominique Culioli, on September 23\rd 1982; her father was a pharmacist from Vico (a village north of Ajaccio, Corsica). They had two sons, Pierre (born in 1985), now a hip-hop producer, and Jean (born in 1986) now a regional councillor in the city of Neuilly-sur-Seine, France. Sarkozy's best man was the prominent right-wing politician Charles Pasqua, later to become a political opponent. Sarkozy divorced Culioli in 1996, although they had already been separated for several years.

\subsubsection*{Cécilia Ciganer-Albéniz}

As mayor of Neuilly-sur-Seine, Sarkozy met former fashion model and public relations executive Cécilia Ciganer-Albéniz (great-granddaughter of composer Isaac Albéniz and daughter of a Moldovan father), when he officiated at her wedding to television host Jacques Martin. In 1988, she left her husband for Sarkozy, and divorced Martin one year later. Sarkozy married her in October 1996, with witnesses Martin Bouygues and Bernard Arnault. They have one son, Louis, born April 23\rd 1997.

Between 2002 and 2005, the couple often appeared together on public occasions, with Cécilia Sarkozy acting as the chief aide for her husband. On May 25\theu 2005, however, the Swiss newspaper Le Matin revealed that she had left Sarkozy for French-Moroccan national Richard Attias, head of Publicis in New York. There were other accusations of a private nature in Le Matin, which led to Sarkozy suing the paper. In the meantime, he was said to have had an affair with a journalist of Le Figaro, Anne Fulda.

Sarkozy and Cécilia ultimately divorced on October 15\theu 2007, soon after his election as President. She was his second wife.

\subsubsection*{Carla Bruni}

Less than a month after separating from Cécilia, Sarkozy met Italian-born singer Carla Bruni at a dinner party, and soon entered a relationship with her. They married on February 2\nd 2008 at the Élysée Palace in Paris.

In 2010, there were controversial reports that the marriage was in trouble. Allegations on Twitter stated that both parties were having extramarital affairs.

\end{document}
