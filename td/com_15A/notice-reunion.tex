\documentclass[12pt]{article}
\usepackage{fontspec}
\usepackage{polyglossia}
\setdefaultlanguage{french}
\usepackage[a4paper,margin=3cm]{geometry}
\usepackage{url,hyperref}
\usepackage{siunitx}
\usepackage{schemabloc}
\usepackage{listings}
\usepackage{auto-pst-pdf}
\usepackage{pst-circ}
\usepackage{soul}
\usepackage{verbatim}
\usepackage{lmodern}
\usepackage{tikz}
\usepackage[european,cuteinductors,siunitx]{circuitikz}
\usepackage{xunicode,xltxtra}
\usepackage{graphicx}
\usepackage{amsmath}
\usepackage{minted}
\usepackage{multicol}
\usepackage{float}
\floatplacement{figure}{H}
\title{\includegraphics{../../../images/inp-enseeiht} \\ ~ \\ ~ \\ ~ \\ ~ \\
Notice technique de conduite d’un réunion}
\author{Guilhem Saurel}
\date{\today}
\renewcommand{\thesection}{\Roman{section}}
\renewcommand{\thesubsection}{\thesection .\alph{subsection}}
\begin{document}

\maketitle

\section{Pré-réunion}
Avant le début de la réunion, un certain nombre de choses doivent être correctement définies:
\begin{itemize}
    \item L’animateur: ce sera la personne qui dirigera une réunion, du début à la fin;
    \item Le secrétaire: ce sera celui qui prendra des notes, pour que les autres puissent se concentrer ser les débats;
    \item Les participants: ils doivent être:
    \begin{itemize}
        \item Compétents;
        \item Mais compétents dans des domaines différents;
        \item Représentant une entitée intéressée par l’objet principal de la réunion;
        \item Responsables auprès de cette entitée des décisions prises par la réunion.
    \end{itemize}
    \item Un crénau horaire qui doit répondre au mieux aux disponibilités des personnes ci-dessus;
    \item Un ordre du jour, qui doit contenir:
    \begin{itemize}
        \item Les différents sujets à apporter;
        \item L’ordre de traitement de ces sujets 
        \item Le temps approximatif de traitement de ces sujets;
        \item Un résumé de la situation sur ces sujets avant la réunion.
    \end{itemize}
\end{itemize}

Une fois que tout ceci est correctement défini, l’animateur (ou le secrétaire), doit envoyer le tout par écrit aux participants: c’est la «convocation».
    
\section{Réunion}

Pendant la réunion, tout participant doit s’efforcer de rester concentré et constructif.

Pour veiller à ceci, l’animateur doit:
\begin{itemize}
    \item S’assurer que tout le monde sais pourquoi il y a une réunion et a bien pris connassancess de tous les points de la convocation;
    \item Veiller à suivre l’ordre du jour;
    \item Prendre en temps réel les décisions sur d’éventuels traitement de sujet proches mais pas à l’ordredu jour;
    \item Recentrer le débat quand il le faut;
    \item Tenir les délais;
    \item Distribuer équitablement la parole;
    \item Interroger les personnes compétentes sur chaque question et prendre en compte tous les points de vue;
    \item Faire en sorte que tout le monde reste respecteux.
\end{itemize}

Enfin, un effort minimum est demandé à chacun:
\begin{itemize}
    \item Arriver a l’heure;
    \item Rester courtois;
    \item Éteindre son portable.
\end{itemize}

\section{Post-réunion}

Le secrétaire envoie enfin un récapitulatif de la réunion à tout le monde.

\end{document}
